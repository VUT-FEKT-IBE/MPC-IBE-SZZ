\section{Formální definice kryptografického systému, symetrické a asymetrické šifry. Výpočetně těžké matematické problémy pro~asymetrické šifry.}

\subsection{Definice}

\uline{Kryptografický systém} pro~šifrování zpráv je pětice $(\mathcal{M}, \mathcal{C}, \mathcal{K}, \mathcal{E}, \mathcal{D})$, kde
$\mathcal{M}$ je prostor otevřených zpráv,
$\mathcal{C}$ prostor šifrových zpráv,
$\mathcal{K}$ prostor klíčů,
$\mathcal{E}, \mathcal{D}$ dvojice zobrazení, které každému klíči $k \in \mathcal{K}$ přiřazují transformaci pro~zašifrování zpráv $E$ a~transformaci pro~dešifrování zpráv $D$, kde platí $D(k(E(k,m))=m \ \forall \ k \in \mathcal{K}, m \in \mathcal{M}$.

\uline{Symetrická šifra} je taková šifra, kde pro~každé $k \in \mathcal{K}$ lze z~transformace zašifrování $E_k$ určit transformaci dešifrování $D_k$ a naopak.

\uline{Asymetrická šifra} je taková šifra, kde pro~skoro všechna $k \in \mathcal{K}$ nelze z~transformace pro~zašifrování $E_k$ určit transformaci dešifrování $D_k$.
Bývá zde přítomen tajný klíč $k$, ze~kterého se vhodnou transformací $G$ vygeneruje dvojice parametrů $(e, d)$, která tvoří veřejné a~privátní klíče ($k_\text{pub}$, $k_\text{priv}$).
Ty parametrizují transformace šifrování a dešifrování.

\subsection{Matematické problémy asymetrických šifer}

\subsubsection{Problém diskrétního logaritmu}

Rovnice $c \equiv m^n \mod p$ je výpočetně jednoduchá a rychlá, získání $m$ z~$c$ je naopak složité.

DLP využívají protokoly Diffie-Hellman, ElGamal, DSA, ECDL, ECDSA, \dots

Mezi algoritmy řešení patří brute-force, baby step--giant step, Pollardův $\rho$ algoritmus, funkční síto, \dots

\subsubsection{Problém faktorizace}

Faktorizace je proces převodu složeného čísla na~jeho prvočíselné složky.

FP využívá například RSA.

Mezi algoritmy řešení patří brute-force, Pollardůvo $\rho$ algoritmus, Pollardův $\rho - 1$ algoritmus, Lehmannova metoda, kvadratické síto, \dots


\clearpage
\section{Služby bezpečnosti zajišťované kryptografickými prostředky, kryptografické mechanismy, které tyto služby zajišťují.}

\clearpage
\section{Kryptograficky bezpečné generátory náhodných čísel –- požadavky, hodnocení bezpečnosti, příklady realizace.}

\clearpage
\section{Hašovací funkce -- požadavky, hodnocení bezpečnosti. Princip konstrukce -- iterační, typu \enquote{houba} (SHA3).}

\clearpage
\section{Kvantový přenos informace -- důvody použití, příklady protokolů.}

\clearpage
\section{Postkvantová kryptografie –- důvody použití, jaké těžké matematické problémy se zde využívají (kryptosystém McEliece, kryptosystém založený na mřížkách). Jednorázový podpis pomocí hašovacích funkcí (Lamport).}

\clearpage
\section{V~souvislosti s~nařízením eIDAS vysvětlete pojmy -- elektronický podpis, zaručený elektronický podpis a kvalifikovaný elektronický podpis, elektronická pečeť, elektronické časové razítko.}

\clearpage
\section{Technologie blockchain -– struktura, princip, možnosti využití.}

\clearpage
\section{Fyzicky neklonovatelné funkce (FNF) -- k~čemu lze využít, výhody a nevýhody, požadované vlastnosti, příklady.}

\clearpage
\section{Autentizační protokoly –- na jakém principu pracují, využívané proměnné parametry, hodnocení jejich bezpečnosti (BAN logika).}
