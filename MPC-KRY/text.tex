\section{Formální definice kryptografického systému, symetrické a asymetrické šifry. Výpočetně těžké matematické problémy pro~asymetrické šifry.}

\subsection{Definice}

\uline{Kryptografický systém} pro~šifrování zpráv je pětice $(\mathcal{M}, \mathcal{C}, \mathcal{K}, \mathcal{E}, \mathcal{D})$, kde
$\mathcal{M}$ je prostor otevřených zpráv,
$\mathcal{C}$ prostor šifrových zpráv,
$\mathcal{K}$ prostor klíčů,
$\mathcal{E}, \mathcal{D}$ dvojice zobrazení, které každému klíči $k \in \mathcal{K}$ přiřazují transformaci pro~zašifrování zpráv $E$ a~transformaci pro~dešifrování zpráv $D$, kde platí $D(k(E(k,m))=m \ \forall \ k \in \mathcal{K}, m \in \mathcal{M}$.

\uline{Symetrická šifra} je taková šifra, kde pro~každé $k \in \mathcal{K}$ lze z~transformace zašifrování $E_k$ určit transformaci dešifrování $D_k$ a naopak.

\uline{Asymetrická šifra} je taková šifra, kde pro~skoro všechna $k \in \mathcal{K}$ nelze z~transformace pro~zašifrování $E_k$ určit transformaci dešifrování $D_k$.
Bývá zde přítomen tajný klíč $k$, ze~kterého se vhodnou transformací $G$ vygeneruje dvojice parametrů $(e, d)$, která tvoří veřejné a~privátní klíče ($k_\text{pub}$, $k_\text{priv}$).
Ty parametrizují transformace šifrování a dešifrování.

\subsection{Matematické problémy asymetrických šifer}

\subsubsection{Problém diskrétního logaritmu}

Rovnice $c \equiv m^n \mod p$ je výpočetně jednoduchá a rychlá, získání $m$ z~$c$ je naopak složité.

DLP využívají protokoly Diffie-Hellman, ElGamal, DSA, ECDL, ECDSA, \dots

Mezi algoritmy řešení patří brute-force, baby step--giant step, Pollardův $\rho$ algoritmus, funkční síto, \dots

\subsubsection{Problém faktorizace}

Faktorizace je proces převodu složeného čísla na~jeho prvočíselné složky.

FP využívá například RSA.

Mezi algoritmy řešení patří brute-force, Pollardůvo $\rho$ algoritmus, Pollardův $\rho - 1$ algoritmus, Lehmannova metoda, kvadratické síto, \dots


\clearpage
\section{Služby bezpečnosti zajišťované kryptografickými prostředky, kryptografické mechanismy, které tyto služby zajišťují.}

\subsection{Služby}

\uline{Autentizace} (\emph{authentication}) je proces ověření identity entity.
\emph{Peer Entity Authentication} ověřuje uživatele,
\emph{Data Origin Authentication} ověřuje všechna data a eliminuje např. útoky opakováním.

\uline{Řízení přístupu} (\emph{access control}) je možnost povolit či odepřít použití určitého zdroje určitému subjektu.

\uline{Zabezpečení důvěrnosti dat} (\emph{data confidentiality}) je ochrana obsahu proti analýze.
Může jít o~zajištění důvěrnosti přenosu zpráv, spojení, toku dat nebo služby selektivní důvěrnosti (které chrání pouze část informace).

\uline{Zabezpečení integrity dat} (\emph{data integrity}) je ochrana proti neautorizované modifikaci.
Slabá integrita: modifikace zprávy šumem, změna pořadí paketů, náhodná duplicita (kontrolní součet, CRC, pořadové číslo).
Silná integrita: podvržení zprávy, úmyslné pozměnění zprávy; bez~oprav a s~opravami.

\uline{Ochrana proti odmítnutí původu zprávy} (\emph{non-repudiation}) zajišťuje důkaz o~původu dat a dokazuje původ nebo doručení.

\subsection{Mechanismy}

Šifrování, digitální podpisy, řízení přístupu, mechanismy integrity dat, výměna autentizační informace, padding, řízení směrování, ověření třetí stranou.

\begin{table}[ht]
	\centering
	\onehalfspacing

	\begin{tabular}{|l|cccccccc|}
		&
		\begin{sideways}šifrování\end{sideways} &
		\begin{sideways}podpis\end{sideways} &
		\begin{sideways}řízení přístupu\end{sideways} &
		\begin{sideways}integrita\end{sideways} &
		\begin{sideways}autentizace\end{sideways} &
		\begin{sideways}padding\end{sideways} &
		\begin{sideways}řízení směrování\end{sideways} &
		\begin{sideways}ověření třetí stranou\end{sideways} \\
		\hline\hline
		%                              š   p   ř   i   a   p   ř   o
		autentizace spojení          & X & X &   &   & X &   &   &   \\
		autentizace odesílatele      & X & X &   &   &   &   &   &   \\
		řízení přístupu              &   &   & X &   &   &   &   &   \\
		důvěrnost spojení            & X &   &   &   &   &   &   &   \\
		důvěrnost přenosu zpráv      & X &   &   &   &   &   & X &   \\
		selektivní důvěrnost         & X &   &   &   &   &   & X &   \\
		důvěrnost toku dat           & X &   &   &   &   & X & X &   \\
		integrita spojení s~opravou  & X &   &   & X &   &   &   &   \\
		integrita spojení bez~opravy & X &   &   & X &   &   &   &   \\
		selektivní integrita spojení & X & X &   & X &   &   &   &   \\
		integrita přenosu zpráv      & X & X &   & X &   &   &   &   \\
		nepopiratelnost odesílatele  & X & X &   & X &   &   &   & X \\
		nepopiratelnost doručení     & X & X &   & X &   &   &   & X \\
	\end{tabular}

	\caption{Matice mechanismů bezpečnosti}
\end{table}


\clearpage
\section{Kryptograficky bezpečné generátory náhodných čísel –- požadavky, hodnocení bezpečnosti, příklady realizace.}

\clearpage
\section{Hašovací funkce -- požadavky, hodnocení bezpečnosti. Princip konstrukce -- iterační, typu \enquote{houba} (SHA3).}

\clearpage
\section{Kvantový přenos informace -- důvody použití, příklady protokolů.}

\clearpage
\section{Postkvantová kryptografie –- důvody použití, jaké těžké matematické problémy se zde využívají (kryptosystém McEliece, kryptosystém založený na mřížkách). Jednorázový podpis pomocí hašovacích funkcí (Lamport).}

\clearpage
\section{V~souvislosti s~nařízením eIDAS vysvětlete pojmy -- elektronický podpis, zaručený elektronický podpis a kvalifikovaný elektronický podpis, elektronická pečeť, elektronické časové razítko.}

\clearpage
\section{Technologie blockchain -– struktura, princip, možnosti využití.}

\clearpage
\section{Fyzicky neklonovatelné funkce (FNF) -- k~čemu lze využít, výhody a nevýhody, požadované vlastnosti, příklady.}

\clearpage
\section{Autentizační protokoly –- na jakém principu pracují, využívané proměnné parametry, hodnocení jejich bezpečnosti (BAN logika).}
\subsection{Princip}
\begin{itemize}
    \item výzva - odpověď
    \item mezi dvěma entitami nebo s využitím třetí důvěryhodné strany
    \item ověřují správnost a čerstvost autentizačního požadavku
    \item výzva musí být čerstvá - využívají náhodná čísla, sekvenční čísla nebo časová razítka
\end{itemize}

\subsection{Využívané proměnné parametry}
\begin{itemize}
    \item sekvenční číslo
    \begin{itemize}
        \item tajné, nutné ukládat poslední použité číslo
        \item při každém použití se sekvenční číslo inkrementuje o 1
        \item v případě desynchronizace je nutné využít nějaký autentizační protokol k synchronizaci
    \end{itemize}
    \item časové razítko
    \begin{itemize}
        \item využívá se maximálního povoleného zpoždění (acceptance-window) přijaté zprávy
        \item přijatá časová razítka jsou ukládána pro případ, kdyby útočník chtěl provést útok přehráním v povoleném časovém okně nebo v případě změny hodin u ověřovatele
    \end{itemize}
    \item náhodné číslo použitelné pouze jednou (nonce = number used only once)
    \begin{itemize}
        \item nevyžaduje synchronizaci
        \item ověřovatel ho zašle protstraně a ta jej použije v autentizační odpovědi
        \item po použití je vyřazeno z databáze
        \item pokud je nonce dostatečně velký (např. 128 b), tak náhodný výběr snižuje pravděpodobnost výběru stejného nonce na zanedbatelnou úroveň
    \end{itemize}
\end{itemize}

\subsection{Hodnocení bezpečnosti - BAN logika}
\begin{itemize}
    \item BAN logika - slouží k formálnímu popisu bezpečnosti autentizačním protokolů
    \item jedna z prvních a nejpoužívanějších logik pro formální ohodnocení autentizačních protokolů, určena pro kryptografii se sdíleným i veřejným klíčem
    \item jedná se o epistemickou a doxastickou logiku (poddruhy modální logiky zabývající se úvahami a znalosti a víře) - logiky využívané v PC vědě a umělé inteligenci
    \item zabývá se autentizačními protokoly na abstrajtní úrovni (neřeší konkrétní implementaci zkoumaného protokolu a problémy s tím spjaté) \\
\end{itemize}

Otázky BAN logiky
\begin{itemize}
    \item čeho chce zkoumaný protokol dosáhnout?
    \item potřebuje zkoumaný protokol více předpokladů než jiný protokol?
    \item vykonává zkoumaný protokol cokoliv nepotřebného, jenž by mohlo být vypuštěno bez ohrožení bezpečnosti?
    \item šifruje zkoumaný protokol něco, co by mohlo být zasláno v otevřené formě bez ohrožení bezpečnosti? \\
\end{itemize}

Definované konstrukce BAN logiky
\begin{itemize}
    \item P věří X
    \begin{itemize}
        \item účastník P věří výroku X, nebo by měl být oprávněný věřit X
        \item účastník P může jednat, jako by X bylo pravdivé
        \item hlavní konstrukce logiky
    \end{itemize}
    \item P vidí X
    \begin{itemize}
        \item účastník P přijal zprávu obsahujcí výrok X, jenž může přečíst a zopakovat X (např. po dešifrování)
    \end{itemize}
    \item P vyslovil (jednou řekl) X
    \begin{itemize}
        \item účastník P v určitém čase zaslal zprávu obsahující výrok X. 
        \item není známo, zda-li byla zpráva odeslána před dlouhou dobou nebo v průběu současného běhu protokolu
        \item je však známo, že účastník P věřil X, když odesílal zprávu \\
    \end{itemize}
    \item pravidlo význam zprávy (message meaning)
    \begin{itemize}
        \item tato pravidla popisují, jakým způsobem lze odvodit důvěry o původu zpráv
    \end{itemize}
    \item pravidlo ověření aktuálnosti zprávy (nonce verification)
    \begin{itemize}
        \item využívá se ke kontrole aktuálnosti (novosti) zprávy 
        \item příjemce může předpokládat, že odesílatel věří jejímu obsahu a můžeme mu také věřit
        \item zajišťuje ochranu proti útoku přehráním
    \end{itemize}
    \item pravidlo novosti celého výroku
    \begin{itemize}
        \item pokud je část výroku nová (fresh), pak je celý výrok nový
    \end{itemize}
    \item pravidlo jurisdikce (jurisdiction)
    \item pravidla důvěry k množině výroků
\end{itemize}
