\section{Formální definice kryptografického systému, symetrické a asymetrické šifry. Výpočetně těžké matematické problémy pro~asymetrické šifry.}

\subsection{Definice}

\uline{Kryptografický systém} pro~šifrování zpráv je pětice $(\mathcal{M}, \mathcal{C}, \mathcal{K}, \mathcal{E}, \mathcal{D})$, kde
$\mathcal{M}$ je prostor otevřených zpráv,
$\mathcal{C}$ prostor šifrových zpráv,
$\mathcal{K}$ prostor klíčů,
$\mathcal{E}, \mathcal{D}$ dvojice zobrazení, které každému klíči $k \in \mathcal{K}$ přiřazují transformaci pro~zašifrování zpráv $E$ a~transformaci pro~dešifrování zpráv $D$, kde platí $D(k(E(k,m))=m \ \forall \ k \in \mathcal{K}, m \in \mathcal{M}$.

\uline{Symetrická šifra} je taková šifra, kde pro~každé $k \in \mathcal{K}$ lze z~transformace zašifrování $E_k$ určit transformaci dešifrování $D_k$ a naopak.

\uline{Asymetrická šifra} je taková šifra, kde pro~skoro všechna $k \in \mathcal{K}$ nelze z~transformace pro~zašifrování $E_k$ určit transformaci dešifrování $D_k$.
Bývá zde přítomen tajný klíč $k$, ze~kterého se vhodnou transformací $G$ vygeneruje dvojice parametrů $(e, d)$, která tvoří veřejné a~privátní klíče ($k_\text{pub}$, $k_\text{priv}$).
Ty parametrizují transformace šifrování a dešifrování.

\subsection{Matematické problémy asymetrických šifer}

\subsubsection{Problém diskrétního logaritmu}

Rovnice $c \equiv m^n \mod p$ je výpočetně jednoduchá a rychlá, získání $m$ z~$c$ je naopak složité.

DLP využívají protokoly Diffie-Hellman, ElGamal, DSA, ECDL, ECDSA, \dots

Mezi algoritmy řešení patří brute-force, baby step--giant step, Pollardův $\rho$ algoritmus, funkční síto, \dots

\subsubsection{Problém faktorizace}

Faktorizace je proces převodu složeného čísla na~jeho prvočíselné složky.

FP využívá například RSA.

Mezi algoritmy řešení patří brute-force, Pollardůvo $\rho$ algoritmus, Pollardův $\rho - 1$ algoritmus, Lehmannova metoda, kvadratické síto, \dots


\clearpage
\section{Služby bezpečnosti zajišťované kryptografickými prostředky, kryptografické mechanismy, které tyto služby zajišťují.}

\subsection{Služby}

\uline{Autentizace} (\emph{authentication}) je proces ověření identity entity.
\emph{Peer Entity Authentication} ověřuje uživatele,
\emph{Data Origin Authentication} ověřuje všechna data a eliminuje např. útoky opakováním.

\uline{Řízení přístupu} (\emph{access control}) je možnost povolit či odepřít použití určitého zdroje určitému subjektu.

\uline{Zabezpečení důvěrnosti dat} (\emph{data confidentiality}) je ochrana obsahu proti analýze.
Může jít o~zajištění důvěrnosti přenosu zpráv, spojení, toku dat nebo služby selektivní důvěrnosti (které chrání pouze část informace).

\uline{Zabezpečení integrity dat} (\emph{data integrity}) je ochrana proti neautorizované modifikaci.
Slabá integrita: modifikace zprávy šumem, změna pořadí paketů, náhodná duplicita (kontrolní součet, CRC, pořadové číslo).
Silná integrita: podvržení zprávy, úmyslné pozměnění zprávy; bez~oprav a s~opravami.

\uline{Ochrana proti odmítnutí původu zprávy} (\emph{non-repudiation}) zajišťuje důkaz o~původu dat a dokazuje původ nebo doručení.

\subsection{Mechanismy}

Šifrování, digitální podpisy, řízení přístupu, mechanismy integrity dat, výměna autentizační informace, padding, řízení směrování, ověření třetí stranou.

\begin{table}[ht]
	\centering
	\onehalfspacing

	\begin{tabular}{|l|cccccccc|}
		&
		\begin{sideways}šifrování\end{sideways} &
		\begin{sideways}podpis\end{sideways} &
		\begin{sideways}řízení přístupu\end{sideways} &
		\begin{sideways}integrita\end{sideways} &
		\begin{sideways}autentizace\end{sideways} &
		\begin{sideways}padding\end{sideways} &
		\begin{sideways}řízení směrování\end{sideways} &
		\begin{sideways}ověření třetí stranou\end{sideways} \\
		\hline\hline
		%                              š   p   ř   i   a   p   ř   o
		autentizace spojení          & X & X &   &   & X &   &   &   \\
		autentizace odesílatele      & X & X &   &   &   &   &   &   \\
		řízení přístupu              &   &   & X &   &   &   &   &   \\
		důvěrnost spojení            & X &   &   &   &   &   &   &   \\
		důvěrnost přenosu zpráv      & X &   &   &   &   &   & X &   \\
		selektivní důvěrnost         & X &   &   &   &   &   & X &   \\
		důvěrnost toku dat           & X &   &   &   &   & X & X &   \\
		integrita spojení s~opravou  & X &   &   & X &   &   &   &   \\
		integrita spojení bez~opravy & X &   &   & X &   &   &   &   \\
		selektivní integrita spojení & X & X &   & X &   &   &   &   \\
		integrita přenosu zpráv      & X & X &   & X &   &   &   &   \\
		nepopiratelnost odesílatele  & X & X &   & X &   &   &   & X \\
		nepopiratelnost doručení     & X & X &   & X &   &   &   & X \\
	\end{tabular}

	\caption{Matice mechanismů bezpečnosti}
\end{table}


\clearpage
\section{Kryptograficky bezpečné generátory náhodných čísel –- požadavky, hodnocení bezpečnosti, příklady realizace.}
\subsection{Požadavky}
\begin{itemize}
    \item \uline{"Next-bit test"} - je-li známo prvních k bitů náhodné posloupnosti, neexistuje žádný algoritmus s polynomiální složitostí, který by dokázal předpovědět (k + 1) bit s pravděpodobností vyšší než 1/2
    \item \uline{"State compromise"} - i když je zjištěn vnitřní stav generátoru (celý nebo z části), nelze zpětně zrekonstruovat dosavadní vygenerovanou posloupnost. Navíc, pokud do generátoru za běhu vstupuje další entrope, nemělo by být možné ze znalosti vnitřního stavu předpovědět vnitřní stav v následujících iteracích
    \item vetšina používaných PRNG tyto požadavky splňuje jen za určitých podmínek
\end{itemize}

\subsection{Hodnocení bezpečnosti}
Entropie
\begin{itemize}
    \item veličina entropie popisuje míru náhodnosti - jak obtížné je hodnotu (náhodné číslo, náhodnou posloupnost, řetězec náhodných bitů) uhodnout
    \item míra nejistoty (nepředvídatelnosti) hodnoty a závisí na pravděpodobnostech možných výsledků procesu, který ji generuje
    \item dána vztahem:
\end{itemize}
\begin{align*}
    H(X) = - \displaystyle\sum\limits_{i-1}^n p_{i} \log_2 p_{i} 
\end{align*}
\begin{itemize}
    \item X - generovaná hodnota
    \item $p_{1},..., p_{n}$ - pravděpodobnosti všech hodnot $X_{1},..., X_{n}$, které je daný generátor schopen vygenerovat
    \item vztahuje se k útočníkovu a jeho schopnosti předpovídat vygenerovanou hodnotu - pokud útočník následující generovanou hodnotu s jistotou zná, entropie je nulová = nulová bezpečnosti aplikace, která takto vygenerovanou náhodnou hodnotu využívá
    \item vyjadřuje průměrný počet bitů nutných k zakódování hodnoty při použití optimálního k=odování
    \item vyjadřuje obsažené množství informace vyjádřené v bitech
    \item entropie generátoru je maximální, pokud se pro danou délku (počet bitů) generují všechny možné posloupnosti, každá z nich se stejnou pravděpodobností
\end{itemize}

\subsection{Příklady realizace}
Generátor pseudonáhodných čísel (PRNG)
\begin{itemize}
    \item algoritmicky řešení generátory, využívají výpočetních / SW metod
    \item základní problém - jakýkoliv v současnosti existující program je deterministický = existuje pouze jedna možnost, jak k výsledku dojít
    \item možnost se může jevit náhodná, ale není
    \item pro člověka pouze není snadné zjistit, jakým způsobem byla vytvořena
    \item výhody: rychlé, snadno realizovatelné, lze nastavit odchylku rozložení
    \item nevýhody: malá bezpečnost, periodicita \\
\end{itemize}

Kryptograficky bezpečené PRNG
\begin{itemize}
    \item pokud PRNG splňuje určité požadavky $\longrightarrow$ považujeme na kryptograficky bezpečné
    \item vyžadují náhodný a tajný vstup (seed) - nelze se tedy obejít bez "skutečné náhody"
    \item kvalita závisí na kvalitě generování hodnoty seed
    \item entropie výstupu PRNG je dána entropií, která vstupuje (seed), algoritmus samotný nikdy nemůže entropii zvyšovat
    \item zdroj entropie např.: sběr událostí z pohybu myši, klávesnice, HD a některých přerušení (Linux PRNG)
    \item podmínky:
    \begin{itemize}
        \item Next-bit test
        \item State compromise
    \end{itemize}
    \item příklady:
    \begin{itemize}
        \item bloková šifra v režimu čítače - náhodně se zvolí klíč / seed a počáteční hodnota čítače i. Zvoleným klíčem se postupně šifrují hodnoty i, i+1, atd. dokud nedojde k překročení velikosti bloku - perioda
        \item hashovací funkce aplikovaná na čítač - náhodně se zvolí počáteční hodnota čítače i. Postupně se  hashují hodnoty i, i+1 atd. Nesmí dojít k prozrazení počáteční hodnoty čítače
        \item proudové šifry - v zásadě PRNG, s jehož výstupem se provádí XOR s otevřeným textem
        \item algoritmy založené na teorii čísel - u kterých byl proveden důkaz bezpečnosti \\
    \end{itemize}
\end{itemize}

Generátory skutečně náhodných čísel (TRNG)
\begin{itemize}
    \item využívající fyzikální / HW metody
    \item zdroj entropie - klasický nebo kvantový fyzikální jev
    \item výhody: vysoká bezpečnost, neopakovatelnost
    \item nevýhody: pomalé, problémy s realizací, ne vždy lze dosáhnout maximálně rovnoměrného rozložení
\end{itemize}



\clearpage
\section{Hašovací funkce -- požadavky, hodnocení bezpečnosti. Princip konstrukce -- iterační, typu \enquote{houba} (SHA3).}
\begin{itemize}
    \item hašovací funkce je matematická funkce (resp. algoritmus) pro převod vstupních dat do malého čísla
    \item formálně jde o funkci h, která převádí vstupní posloupnost bitů na posloupnost pevné délky n bitů
    \item použití:
    \begin{itemize}
        \item tvorba digitálních podpisů - otisky zprávy
        \item kontrola integrity, porovnání souborů - otisk souboru
        \item jednoznačná identifikace dat
        \item uložení hesel - otisky hesel
        \item uložení klíčů - otisky klíčů
        \item prokazování autorství
        \item pseudonáhodné generátory \\
    \end{itemize}
\end{itemize}

\subsection{Požadavky}
\begin{itemize}
    \item k "libovolně" velkému vstupu M - pevná délka výstupu h
    \item jednocestnost (one-way)
    \begin{itemize}
        \item pro danou zprávu M lze snadno spočítat h = h(M)
        \item je-li dáno h, je velmi těžké spočítat M, ne však nemožné
    \end{itemize}
    \item bezkoliznost (collision-free)
    \begin{itemize}
        \item je velmi těžké nalézt různá M a M' tak, aby h(M) = h(M') \\
    \end{itemize}
\end{itemize}

Odolnost proti kolizím
\begin{itemize}
    \item kolize zákonitě existují, ale je oblížné je nalézt
\end{itemize}
\begin{enumerate}
    \item kolize prvního řádu (preimage resistance)
    \begin{itemize}
        \item nalezení vzoru
    \end{itemize}
    \item kolize druhého řádu (second preimage resistance)
    \begin{itemize}
        \item k dané zprávě $M_{1}$ nalézt zprávu $M_{2}$ tak, aby platilo $h(M_{1}) = h(M_{2})$
    \end{itemize}
    \item odolnost vůči kolizím (collistion resistance)
    \begin{itemize}
        \item nalezení dvou libovolných zpráv M a M', pro které platí h(M) = h(M') \\
    \end{itemize}
\end{enumerate}

\subsection{Hodnocení bezpečnosti}
Útoky, cílem je nalezení kolize
\begin{itemize}
    \item hrubou silou - metoda totálních zkoušek - závisí na délce hašovacího kódu
    \item kryptoanalýzou - zaměřená především na vnitřní strukturu haš. fce, především na kompresní funkci f
\end{itemize}

\subsection{Princip konstrukce - iterační}
\begin{itemize}
    \item tento princip používají drtivá většina dnes používaných hashovacích funkcí
    \item algoritmus je založený na opakovaném použití kompresní funkce f (= je jádrem hashovacích funkcí)
    \item proces iteračního výpočtu hashového kódu h je možné vyjádřit v následujícím tvaru:
\end{itemize}
\begin{align*}
    IV &= h_{0} \\
    h_{i} &= f (h_{i-1}, M_{i}), pro 1 \leq i \leq n \\
    h &= h_{n} \\
\end{align*}
\begin{itemize}
    \item tzn. kompresní funkce má dva vstupy a jeden výstup
    \item ypracovává aktuální blok zprávy $M_{i}$ a hodnotu $h_{i-1}$
    \item Výstupem je určitá hodnota $h_{i-1}$ (= kontext) $\longrightarrow$ ten pak použit jako vstup do kompresní funkce v dalším kroku
    \item počáteční hodnota kontextu $h_{0}$ - inicializační vektor IV - určuje blok bitů vstupující do první kompresní funkce a zahajuje tak vlastní hashování
\end{itemize}

\subsection{Pricip konstrukce - houba (SHA3)}
\begin{itemize}
    \item kryptografické primitivum využívající se v mnoha aplikacích
    \item dvě fáze: absorpce a vymačkávání
    \item původní zpráva se po bitech absorbuje v první části pomocí kompresních funkcí f
    \item následně je vymačkávána tolikrát, dokud nemáme odpovídající velikost výstupu
    \item příklady: SHA-3 (algoritmus Keccak)
    \begin{itemize}
        \item SHA-3 hash - digitální podpis, integrita dat, identifikace dat
        \item SHA-3 MGF - generování klíčů, hashování v PKI
        \item SHA-3 solená hash - ukládání hesel
        \item SHA-3 MAC - jednodušší než HMAC
        \item SHA-3 proudová šifra
    \end{itemize}
\end{itemize}


\clearpage
\section{Kvantový přenos informace -- důvody použití, příklady protokolů.}

\clearpage
\section{Postkvantová kryptografie –- důvody použití, jaké těžké matematické problémy se zde využívají (kryptosystém McEliece, kryptosystém založený na mřížkách). Jednorázový podpis pomocí hašovacích funkcí (Lamport).}

\clearpage
\section{V~souvislosti s~nařízením eIDAS vysvětlete pojmy -- elektronický podpis, zaručený elektronický podpis a kvalifikovaný elektronický podpis, elektronická pečeť, elektronické časové razítko.}
\subsection{Elektronický podpis}
\begin{itemize}
    \item lze tak označit cokoliv, co je použito  jako podpis dané osoby a co má elektronikcou podobu
    \item např. napsání našeho jména na konec mailu
    \item je zřejmé, ž není zaručeno jednoznačné spojení s podepisující osobou
\end{itemize}

\subsection{Zaručený elektronický podpis}
\begin{itemize}
    \item musí být jednoznačně spojen s podepisující osobou a musí umožňovat její identifikaci
    \item musí být vytvořen pomocí služeb pro vytváření elektronických podpisů - pomocí certifikátu (na ten ale nejsou kladeny žádné požadavky)
    \item nemusí se jednat o certifikát vydaný kvalifikovaným poskytovatelem, může být jakýkoliv, i vystavený svépomocí
\end{itemize}

\subsection{Kvalifikovaný elektronický podpis}
\begin{itemize}
    \item zaručený elektronický podpis vytvořený kvalifikovaným prostředkem pro vytváření elektronických podpisů a založen na kvalifikovaném certifikátu pro elektronické podpisy
    \item kvalifikovaný certifikát = vydaný kvalifikovaných poskytovatelem služeb vytvořejícíh důvěru, tzn. poskytovatel, kterému orgán dohledu udělil status kvalifikovaného poskytovatele (v ČR v současnosti 3)
\end{itemize}

\subsection{Elektronická pečeť}
\begin{itemize}
    \item vydává se jen právnických osobám
    \item p. o. nemůže pečetí opatřit cokoliv, ale jen to, čeho je původcem
    \item proces pečetění totožný s podepisováním elektronických podpisem
\end{itemize}

\subsection{Elektronické časové razítko}
\begin{itemize}
    \item elektronický ekvivalent časového určení a místa vlastního podpisu na listině
    \item elektronický podpis dle znění zákona tento problém neřeší
    \item řeší možné problémy vzniklé odvoláním certifikátu - byl el. dokument podepsán před odvoláním?
    \item zajišťuje důkaz o existenci dokumentu v daném čase
    \item struktura podobná certifikátu, která svazuje kontrolní součet (hash) z dokumentu s časem \\
    \item nutné pro poskytování elektronických notářských služeb a zajištění dlouhodobé archivace elektronicky podepsaných dokumentů
    \item časové razítko je elektronicky podepsáno (vydáváno) autoritou pro vydávání časových razítek - Time Stamping Authority (TSA)
    \item elektronicky podepsaná struktura čas. razítka:
    \begin{itemize}
        \item jméno vydavatele (jméno TSA)
        \item jedinečné sériové číslo razítka
        \item kontrolní součet (hash) z dokumentu a čas \\
    \end{itemize}
\end{itemize}
Požadavky na zdroj času
\begin{itemize}
    \item musí pocházet z oficiálního důvěryhodného zdroje - např. od náhodní časové autority
    \item čas nesmí být možné cestou změnit
    \item vždy musí být možné zpětně dosledovat zdroj času, tedy celou hierarchii časových serverů \\
\end{itemize}
Vydání časového razítka
\begin{itemize}
    \item žádá se prostřednictvím klientské aplikace
    \item klient vytvoří a odešle žádost o časové razítko ve standardizovaném formátu 
    \item žádost je datová struktura obsahující hash z dokumentu
    \item TSA v případě kladné odpovědi odesílá odpověď na žádost obsahující časové razítko
\end{itemize}


\clearpage
\section{Technologie blockchain -– struktura, princip, možnosti využití.}

\clearpage
\section{Fyzicky neklonovatelné funkce (FNF) -- k~čemu lze využít, výhody a nevýhody, požadované vlastnosti, příklady.}

\clearpage
\section{Autentizační protokoly –- na jakém principu pracují, využívané proměnné parametry, hodnocení jejich bezpečnosti (BAN logika).}
