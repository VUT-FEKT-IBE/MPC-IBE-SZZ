\section{Formální definice kryptografického systému, symetrické a asymetrické šifry. Výpočetně těžké matematické problémy pro~asymetrické šifry.}

\subsection{Definice}

\uline{Kryptografický systém} pro~šifrování zpráv je pětice $(\mathcal{M}, \mathcal{C}, \mathcal{K}, \mathcal{E}, \mathcal{D})$, kde
$\mathcal{M}$ je prostor otevřených zpráv,
$\mathcal{C}$ prostor šifrových zpráv,
$\mathcal{K}$ prostor klíčů,
$\mathcal{E}, \mathcal{D}$ dvojice zobrazení, které každému klíči $k \in \mathcal{K}$ přiřazují transformaci pro~zašifrování zpráv $E$ a~transformaci pro~dešifrování zpráv $D$, kde platí $D(k(E(k,m))=m \ \forall \ k \in \mathcal{K}, m \in \mathcal{M}$.

\uline{Symetrická šifra} je taková šifra, kde pro~každé $k \in \mathcal{K}$ lze z~transformace zašifrování $E_k$ určit transformaci dešifrování $D_k$ a naopak.

\uline{Asymetrická šifra} je taková šifra, kde pro~skoro všechna $k \in \mathcal{K}$ nelze z~transformace pro~zašifrování $E_k$ určit transformaci dešifrování $D_k$.
Bývá zde přítomen tajný klíč $k$, ze~kterého se vhodnou transformací $G$ vygeneruje dvojice parametrů $(e, d)$, která tvoří veřejné a~privátní klíče ($k_\text{pub}$, $k_\text{priv}$).
Ty parametrizují transformace šifrování a dešifrování.

\subsection{Matematické problémy asymetrických šifer}

\subsubsection{Problém diskrétního logaritmu}

Rovnice $c \equiv m^n \mod p$ je výpočetně jednoduchá a rychlá, získání $m$ z~$c$ je naopak složité.

DLP využívají protokoly Diffie-Hellman, ElGamal, DSA, ECDL, ECDSA, \dots

Mezi algoritmy řešení patří brute-force, baby step--giant step, Pollardův $\rho$ algoritmus, funkční síto, \dots

\subsubsection{Problém faktorizace}

Faktorizace je proces převodu složeného čísla na~jeho prvočíselné složky.

FP využívá například RSA.

Mezi algoritmy řešení patří brute-force, Pollardůvo $\rho$ algoritmus, Pollardův $\rho - 1$ algoritmus, Lehmannova metoda, kvadratické síto, \dots


\clearpage
\section{Služby bezpečnosti zajišťované kryptografickými prostředky, kryptografické mechanismy, které tyto služby zajišťují.}

\subsection{Služby}

\uline{Autentizace} (\emph{authentication}) je proces ověření identity entity.
\emph{Peer Entity Authentication} ověřuje uživatele,
\emph{Data Origin Authentication} ověřuje všechna data a eliminuje např. útoky opakováním.

\uline{Řízení přístupu} (\emph{access control}) je možnost povolit či odepřít použití určitého zdroje určitému subjektu.

\uline{Zabezpečení důvěrnosti dat} (\emph{data confidentiality}) je ochrana obsahu proti analýze.
Může jít o~zajištění důvěrnosti přenosu zpráv, spojení, toku dat nebo služby selektivní důvěrnosti (které chrání pouze část informace).

\uline{Zabezpečení integrity dat} (\emph{data integrity}) je ochrana proti neautorizované modifikaci.
Slabá integrita: modifikace zprávy šumem, změna pořadí paketů, náhodná duplicita (kontrolní součet, CRC, pořadové číslo).
Silná integrita: podvržení zprávy, úmyslné pozměnění zprávy; bez~oprav a s~opravami.

\uline{Ochrana proti odmítnutí původu zprávy} (\emph{non-repudiation}) zajišťuje důkaz o~původu dat a dokazuje původ nebo doručení.

\subsection{Mechanismy}

Šifrování, digitální podpisy, řízení přístupu, mechanismy integrity dat, výměna autentizační informace, padding, řízení směrování, ověření třetí stranou.

\begin{table}[ht]
	\centering
	\onehalfspacing

	\begin{tabular}{|l|cccccccc|}
		&
		\begin{sideways}šifrování\end{sideways} &
		\begin{sideways}podpis\end{sideways} &
		\begin{sideways}řízení přístupu\end{sideways} &
		\begin{sideways}integrita\end{sideways} &
		\begin{sideways}autentizace\end{sideways} &
		\begin{sideways}padding\end{sideways} &
		\begin{sideways}řízení směrování\end{sideways} &
		\begin{sideways}ověření třetí stranou\end{sideways} \\
		\hline\hline
		%                              š   p   ř   i   a   p   ř   o
		autentizace spojení          & X & X &   &   & X &   &   &   \\
		autentizace odesílatele      & X & X &   &   &   &   &   &   \\
		řízení přístupu              &   &   & X &   &   &   &   &   \\
		důvěrnost spojení            & X &   &   &   &   &   &   &   \\
		důvěrnost přenosu zpráv      & X &   &   &   &   &   & X &   \\
		selektivní důvěrnost         & X &   &   &   &   &   & X &   \\
		důvěrnost toku dat           & X &   &   &   &   & X & X &   \\
		integrita spojení s~opravou  & X &   &   & X &   &   &   &   \\
		integrita spojení bez~opravy & X &   &   & X &   &   &   &   \\
		selektivní integrita spojení & X & X &   & X &   &   &   &   \\
		integrita přenosu zpráv      & X & X &   & X &   &   &   &   \\
		nepopiratelnost odesílatele  & X & X &   & X &   &   &   & X \\
		nepopiratelnost doručení     & X & X &   & X &   &   &   & X \\
	\end{tabular}

	\caption{Matice mechanismů bezpečnosti}
\end{table}


\clearpage
\section{Kryptograficky bezpečné generátory náhodných čísel –- požadavky, hodnocení bezpečnosti, příklady realizace.}

\clearpage
\section{Hašovací funkce -- požadavky, hodnocení bezpečnosti. Princip konstrukce -- iterační, typu \enquote{houba} (SHA3).}

\clearpage
\section{Kvantový přenos informace -- důvody použití, příklady protokolů.}

\clearpage
\section{Postkvantová kryptografie –- důvody použití, jaké těžké matematické problémy se zde využívají (kryptosystém McEliece, kryptosystém založený na mřížkách). Jednorázový podpis pomocí hašovacích funkcí (Lamport).}

\clearpage
\section{V~souvislosti s~nařízením eIDAS vysvětlete pojmy -- elektronický podpis, zaručený elektronický podpis a kvalifikovaný elektronický podpis, elektronická pečeť, elektronické časové razítko.}
\subsection{Elektronický podpis}
\begin{itemize}
    \item lze tak označit cokoliv, co je použito  jako podpis dané osoby a co má elektronikcou podobu
    \item např. napsání našeho jména na konec mailu
    \item je zřejmé, ž není zaručeno jednoznačné spojení s podepisující osobou
\end{itemize}

\subsection{Zaručený elektronický podpis}
\begin{itemize}
    \item musí být jednoznačně spojen s podepisující osobou a musí umožňovat její identifikaci
    \item musí být vytvořen pomocí služeb pro vytváření elektronických podpisů - pomocí certifikátu (na ten ale nejsou kladeny žádné požadavky)
    \item nemusí se jednat o certifikát vydaný kvalifikovaným poskytovatelem, může být jakýkoliv, i vystavený svépomocí
\end{itemize}

\subsection{Kvalifikovaný elektronický podpis}
\begin{itemize}
    \item zaručený elektronický podpis vytvořený kvalifikovaným prostředkem pro vytváření elektronických podpisů a založen na kvalifikovaném certifikátu pro elektronické podpisy
    \item kvalifikovaný certifikát = vydaný kvalifikovaných poskytovatelem služeb vytvořejícíh důvěru, tzn. poskytovatel, kterému orgán dohledu udělil status kvalifikovaného poskytovatele (v ČR v současnosti 3)
\end{itemize}

\subsection{Elektronická pečeť}
\begin{itemize}
    \item vydává se jen právnických osobám
    \item p. o. nemůže pečetí opatřit cokoliv, ale jen to, čeho je původcem
    \item proces pečetění totožný s podepisováním elektronických podpisem
\end{itemize}

\subsection{Elektronické časové razítko}
\begin{itemize}
    \item elektronický ekvivalent časového určení a místa vlastního podpisu na listině
    \item elektronický podpis dle znění zákona tento problém neřeší
    \item řeší možné problémy vzniklé odvoláním certifikátu - byl el. dokument podepsán před odvoláním?
    \item zajišťuje důkaz o existenci dokumentu v daném čase
    \item struktura podobná certifikátu, která svazuje kontrolní součet (hash) z dokumentu s časem \\
    \item nutné pro poskytování elektronických notářských služeb a zajištění dlouhodobé archivace elektronicky podepsaných dokumentů
    \item časové razítko je elektronicky podepsáno (vydáváno) autoritou pro vydávání časových razítek - Time Stamping Authority (TSA)
    \item elektronicky podepsaná struktura čas. razítka:
    \begin{itemize}
        \item jméno vydavatele (jméno TSA)
        \item jedinečné sériové číslo razítka
        \item kontrolní součet (hash) z dokumentu a čas \\
    \end{itemize}
\end{itemize}
Požadavky na zdroj času
\begin{itemize}
    \item musí pocházet z oficiálního důvěryhodného zdroje - např. od náhodní časové autority
    \item čas nesmí být možné cestou změnit
    \item vždy musí být možné zpětně dosledovat zdroj času, tedy celou hierarchii časových serverů \\
\end{itemize}
Vydání časového razítka
\begin{itemize}
    \item žádá se prostřednictvím klientské aplikace
    \item klient vytvoří a odešle žádost o časové razítko ve standardizovaném formátu 
    \item žádost je datová struktura obsahující hash z dokumentu
    \item TSA v případě kladné odpovědi odesílá odpověď na žádost obsahující časové razítko
\end{itemize}


\clearpage
\section{Technologie blockchain -– struktura, princip, možnosti využití.}

\clearpage
\section{Fyzicky neklonovatelné funkce (FNF) -- k~čemu lze využít, výhody a nevýhody, požadované vlastnosti, příklady.}

\clearpage
\section{Autentizační protokoly –- na jakém principu pracují, využívané proměnné parametry, hodnocení jejich bezpečnosti (BAN logika).}
