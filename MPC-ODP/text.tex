\section{Režim omezení odpovědnosti poskytovatelů služeb informační společnosti typu mere conduit, caching a hosting}

{}Relevantní právní předpisy:
\\\href{https://eur-lex.europa.eu/legal-content/CS/ALL/?uri=CELEX:32000L0031
}{Směrnice 2000/31/ES o~některých právních aspektech služeb informační společnosti [\dots]}
\\\href{https://www.zakonyprolidi.cz/cs/2004-480}{Zákon 480/2004 Sb. o~některých službách informační společnosti}

Poskytovatel služeb informační společnosti (ISP) se může nacházet v~jednom ze~tří režimů odpovědnosti v~závislosti na~službách které nabízí: prostý přenos, ukládání v~mezipaměti, shromažďování informací.

Tyto principy vychází především z~faktu, že ISP nemohou za~obsah odpovídat plně (protože by to pro~ně nebylo ekonomicky udržitelné na~pojištění) ani vůbec (protože by stát neměl možnost své zákony vymáhat).

Kdy služba spoluzodpovídá je stanoveno v~autorském zákoně, v~občanském zákoníku a~dalších relevantních zdrojích; Směrnice upravuje kdy poskytovatelé odpovědní nejsou.

\subsection{Mere Conduit (Prostý přenos)}
Patří sem -- poskytovatel připojení (např. wifi VUT).
Poskytovatel není odpovědný, pokud
(a) není původcem přenosu,
(b) nevolí příjemce přenášené informace a
(c) nevolí a~nezmění obsah přenášené informace (data jen přenáší).

Jednotlivé členské státy mohou přikázat konkrétní komunikaci přerušit (v~ČR jde například o~nelicencované hazardní hry online), ale nesmí přikázat prohlížet obsah posílaných správ a tím kontrolovat aktivitu.

\subsection{Caching, mirroring (Ukládání v~mezipaměti)}
Jedná se o ISP, který má 1 server v Amerika a v Európe druhý na cachování.
Poskytovatel není odpovědný, pokud
(a) informaci nezměnil,
(b) vyhovuje podmínkám přístupu k~informaci,
(c) dodržuje pravidla o~aktualizaci informace,
(d) nepřekročí povolené používání technologie obecně uznávané a~používané v~průmyslu s~cílem získat údaje o~užívání a
(e) ihned přijme opatření vedoucí k~odstranění jím uložené informace.

V případe, že jsou na hlavním serveri protiprávní data, je nutno je odstranit i v cachi.

\subsection{Hosting (Shromažďování informací)}

Hosting je služba pracující s~daty generovanými uživateli (facebook, twittter\dots).

Poskytovatel \emph{je} odpovědný, pokud se seznámí s~protiprávním jednáním a~přesto nekoná: jde o~princip \emph{Notice -- Takedown}.
Musí konat okamžitě (\emph{expediciously}) jakmile se o~problematickém stavu dozví%
\footnote{
	V~Německu je \emph{okamžitě} chápáno z~hlediska poskytovatele.
	U~nás se takový případ ještě k~soudu nedostal.
}%
.
V~Itálii a~Španělsku musí protiprávní skutečnost oznámit stát, ve~většině členských států to může udělat kdokoliv.
V~USA to musí být ten jehož práva jsou dotčena.

% TODO Zde je možné zmínit některé příklady:
% Napster ("Ekonomický model by bez autorskoprávně chráněného obsahu nefungoval")
% Delfi AS v. Estonia ("Posuzuje se doba od protiprávního jednání, ne od jeho nahlášení, protože provozovatel musel o urážlivých komentářích vědět.")
% eBay, Uber & Airbnb a jejich umístění na škále zprostředkovatel--poskytovatel

\clearpage
\section{Aktivní povinnosti poskytovatelů služeb informační společnosti (monitoring, filtrování)}
% Snad je to dobře kekw
Není možné chtít dohled nad daty (fyzicky i ekonomicky). Vznikla opatření kdy není odpovědnost (viz 1.). ISP tedy neodpovídá za obsah.
Členské státy nesmí uložit ISP typu mere conduit, ... obecnou povinnost dohlížet na jimi přenášené či ukládané informace nebo obecnou povinnost vyhledávat skutečnosti a okolnosti poukazující na protiprávní činnost. Zákaz se netýká zvláštního dohledu (Rolex vs ricardo.de, dohled nad jedním uživatelem apod.).
Obecný dohled je takový, jenž probíhal:
\begin{itemize}
    \item na všech přenášených či ukládaných datech
    \item vůči všem uživatelům bez rozdílu
    \item preventivně
    \item výlučně na náklady poskytovatele
    \item bez časového omezení
\end{itemize}

Jedná se o zásah do práv ISP (právo na svobodu podnikání) a do práv třetích osob/uživatelů (svoboda projevu a ochrana soukromí).

Česko, článek 15 odst. 1 směrnice
ISP nejsou povinni:
\begin{itemize}
    \item dohlížet na obsah jimi přenášených nebo ukládaných informaci, ani
    \item aktivně vyhledávat skutečnosti a okolnosti poukazující na protiprávní obsah informace
\end{itemize}
Je tedy zakázán jakýkoliv dohled, včetně zvláštního. Dohled jen pokud ISP sám chce.
Dohled je technický (automatizovaný) a manuální (lidský faktor)
Notice-takedown, pokud je ISP upozorněn na protiprávní obsah, musí ho řešit.

\clearpage
\section{Pojem osobního údaje, titul ke zpracování osobních údajů, zvláštní kategorie osobních údajů}

{}Relevantní právní předpisy:
\\\href{https://eur-lex.europa.eu/legal-content/CS/ALL/?uri=CELEX:32016R0679
}{Nařízení 2016/679 o~ochraně fyzických osob v~souvislosti se zpracováním osobních údajů a o volném pohybu těchto údajů [\dots]}
\\\href{https://www.zakonyprolidi.cz/cs/2019-110}{Zákon 110/2019 Sb. o~zpracování osobních údajů}

Osobními údaji jsou veškeré informace o~identifikované/identifikovatelné fyzické osobě, kterou lze přímo či~nepřímo identifikovat.

Jde například o~jméno, identifikační číslo, lokační údaje, síťový identifikátor nebo na jeden či více zvláštních prvků fyzické, fyziologické, genetické, psychické, ekonomické, kulturní nebo společenské identity této fyzické osoby.

\subsection{Zpracování osobních údajů}

Článek 6 vymezuje zákonné tituly -- podmínky, ze~kterých alespoň jedna musí být platná, aby se jednalo o~zpracovnání dle~GDPR:

\begin{enumerate}[label=\alph*)]
\item subjekt udělil souhlas se~zpracováním,
\item zpracování je nezbytné pro~plnění smlouvy,
\item zpracování je nezbytné pro~plnění právní povinnosti správce,
\item zpracování je nezbytné pro~ochranu životně důležitých zájmů subjektu,
\item zpracování je nezbytné pro~splnění úkolu ve~veřejném zájmu nebo výkonu veřejné moci,
\item zpracování je nezbytné pro~účely oprávněných zájmů správce.
\end{enumerate}

Osobní údaje musí být zpracovávány korektně, zákonně a~transparentně.
Musí být shromažďovány pro~určité, výslovně vyjádřené legitimní účely a~nesmí být zpracovávány způsobem, který je s~těmito účely neslučitelný.
Musí být přiměřené, relevantní a~omezené na~nezbytný rozsah ve~vztahu k~účelu pro~který jsou zpracovávány (tzv. \emph{minimalizace údajů}).
Nesmí být zpracovávány po~dobu delší než nezbytnou pro~účely, pro~které jsou zpracovávány.

Zpracování pro~osobní potřebu se pod~Nařízení nevztahuje (viz důvod č.~18).
Pod~tuto kategorii spadají například čísla kontaktů v~telefonu nebo podpisy na~výsledcích tvorby umělecké činnosti%
\footnote{V~době zavádění GDPR se šířily poplašné zprávy, že není možné zveřejňovat autory výkresů v~mateřských školách.}.

\subsection{Zvláštní kategorie osobních údajů}
Např. údaje o zdravotním stavu, informace o očkování -- souhlas na zpracování musí být\textbf{výslovný} (explicitně daný).
Zvláštní kategorie osobních údajů by dle důvodu č.~53 měly být zpracovávány pouze

\begin{itemize}
\item pro~zdravotní účely, je-li jich třeba k~dosáhnutí prospěchu fyzických osob nebo společnosti jako celku,
\item pro~účely monitorování a~varování nebo pro~účely archivace ve~veřejném zájmu,
\item pro~účely vědeckého či~historickéhov výzkumu,
\item pro~statistické účely na~základě práva Unie nebo členského státu,
\item pro~studie prováděné ve~veřejném zájmu v~oblasti veřejného zdraví.
\end{itemize}

Zakazuje se zpracování osobních údajů, které vypovídají o~rasovém či etnickém původu, politických názorech, náboženském vyznání či filozofickém přesvědčení [\dots], a~zpracování genetických údajů, biometrických údajů za~účelem jedinečné identifikace fyzické osoby a~údajů o~zdravotním stavu či o~sexuálním životě nebo sexuální orientaci fyzické osoby.

Tento zákaz neplatí, pokud se uplatní jeden z~následujících případů (viz článek 9):

\begin{enumerate}[label=\alph*)]
\item subjekt údajů udělil výslovný souhlas,
\item zpracování je nezbytné pro~účely plnění povinnosti a~výkon zvláštních práv správce [\dots] v~oblasti pracovního práva nebo práva v~oblasti sociálního zabezpečení [\dots],
\item zpracování je nutné pro~ochranu životně důležitých zájmů subjektu [\dots],
\item zpracování provádí [\dots] nadace, sdružení [\dots] pro~vnitřní účely [\dots],
\item zpracování se týká osobních údajů zjevně zveřejněných subjektem,
\item zpracování je nezbytné pro~určení, výkon nebo obhajobu právních nároků [\dots],
\item zpracování je nezbytné z~důvodu významného veřejného zájmu [\dots],
\item zpracování je nezbytné pro~účely preventivního nebo pracovního lékařství [\dots],
\item zpracování je nezbytné z důvodů veřejného zájmu v oblasti veřejného zdraví [\dots],
\item zpracování je nezbytné pro účely archivace ve veřejném zájmu, pro účely vědeckého či historického výzkumu nebo pro statistické účely [\dots].
\end{enumerate}

\clearpage
\section{Právní postavení správce a zpracovatele osobních údajů}
\subsection*{Správce}
Osoba, která určuje účel zpracování osobních údajů. Musí definovat, jaké údaje, na základě čeho (titul), proč (účel), jako (proces). Všechny informace musí být řádně zdokumentovány.

Soud potvrdil, že i Google je správce osobních údajů.

\subsection*{Zpracovatel}
Subjekt, který je  pověřen správcem na zpracování osobních údajů. V IT se jedná o provozovatele.

\clearpage
\section{Práva subjektů osobních údajů}
Subjekt údajů ==> \uv{člověk}
Práva subjektů:
\begin{itemize}
        \item právo na informace (jaký je účel? jaký je rozsah zpracování?).
        \item přístup, opravu (když jsou údaje chybné).
        \item vymazání (právo byť zapomenutý => i když se dodržuje právní důvod na uchovaní, když to subjekt už více nechce,  je silnější).
        \item omezení zpracování, přenos, námitka.
\end{itemize}

\clearpage
\section{Povinné subjekty dle zákona o kybernetické bezpečnosti}
Kategorie povinných subjektů dle směrnice NIS:

\subsubsection*{Kritická informační infrastruktura} -- systémy, na kterých je závislá informační struktura (český velký kravín, DNS).
\subsubsection*{Významné informační systémy} -- systémy, které kdyby nefungovali, způsobili by významnější problémy (informační systém VŠ).
\subsubsection*{Významné sítě} -- kritérium zahraniční konektivity.
\subsubsection*{Základní služby} -- plynovody se soukromým vlastníkem, nemocnice (nejsou tak velké, aby byli v kritické).
\subsubsection*{Služby a sítě elektronické komunikace} -- mají za povinnost nahlásit kontaktní údaje, plnit direktiva NUKIBU v případe problému.
\subsubsection*{Digitální služby} -- digitální trhoviště, cloudové služby, vyhledávač. Povinnost hlásit fatální incidenty, nemusí zavést kybernetická opatření.

\subsection{Definice v zákoně o kybernetické bezpečnosti}

Orgány a osobami, kterým se ukládají povinnosti v oblasti kybernetické bezpečnosti, jsou

\begin{enumerate}[label=\alph*)]
\item poskytovatel služby elektronických komunikací a subjekt zajišťující síť elektronických komunikací1), pokud není orgánem nebo osobou podle písmene b),
\item orgán nebo osoba zajišťující významnou síť, pokud nejsou správcem nebo provozovatelem komunikačního systému podle písmene d),
\item správce a provozovatel informačního systému kritické informační infrastruktury,
\item správce a provozovatel komunikačního systému kritické informační infrastruktury,
\item správce a provozovatel významného informačního systému,
\item správce a provozovatel informačního systému základní služby, pokud nejsou správcem nebo provozovatelem podle písmene c) nebo d),
\item provozovatel základní služby, pokud není správcem nebo provozovatelem podle písmene f), a
\item poskytovatel digitální služby.
\end{enumerate}
a ještě
Orgán veřejné moci využívající služeb poskytovatelů cloud computingu

\clearpage
\section{Bezpečnostní opatření, varování, reaktivní opatření a ochranná opatření dle zákona o kybernetické bezpečnosti}
Nástroje NUKIBU :
\subsection*{Bezpečnostní opatření}
Standard, který musí zajistit povinný subjekt (pomocí performativních pravidel); proškolení zaměstnanci, řízené vztahy s dodavatelem. Státní orgán nemůže udělit pokutu a nápravu, protože NEEXISTUJE ideální řešení, podle kterého se povinný subjekt musí řídit (je to jeho volba). ASI? :D

\subsection*{Varování}
Nezakládá žádnou povinnost, jenom upozorňuje na bezpečnostní riziko. Když je ale vydané varovaní pro povinné subjekty, subjekt ho ignoroval a vznikne mu škoda s tým související, je za to odpovědný (prevenční povinnost).

\subsection*{Ochranné opatření}
Je vydané na základě zhodnocení bezpečnostního incidentu ==> do budoucnosti, aby se ochránili systémy.

\subsection*{Reaktivní opatření}
Bezprostřední reakce na incident. Rozhodnutí má svého adresáta, musí se ihned vykonat. Může mít vzor OOP (opatření obecné povahy) -- právní předpis neurčitého druhu, např. dopravná značka.

\subsection*{Opatření k nápravě}
Obecný nástroj správného práva, ukládá se když přídě úřad na kontrolu a zjistí nedostatky. Má formu rozhodnutí. Může vstoupit do vztahu mezi ministerstvem (správce) a dodavatelem (ministerstva mají často zle nastavené smlouvy s dodavateli, kteří to lépe zabezpečit nevědí nebo chtějí moc peněz).


\clearpage
\section{Procesní nástroje pro zajištování elektronických důkazů}
\subsection{Zjednodušeně}

K zajišťování se často používají nepříliš vhodné nástroje (zastaralý trestní řád). Může tedy dojít k tomu, že je důkaz nevyužitelný. Důkazy mohou být absolutně a relativně neúčinné.

K počítačovým datům se lze dostam třemi základními způsoby:
\begin{itemize}
    \item Zajištění zařízení, nebo datových nosičů, na kterých jsou počítačová data uchovávána
    \item Získání přímého přístupu k počítačovým datům uchovaným v počítačových systémech
    \item Získání počítačových dat od poskytovatele služeb
\end{itemize}

\subsubsection{Zajištění zařízení, nebo datových nosičů}
Jedna z nejefektivnějších metod získání přístupu k datům. Existuje povinnost vydat věc, pokud je důležitá pro trestní řízení a je ji nutné zajistit. Věc musí vydat ten kdo ji momentálně drží, ne jen majitel. Je nutné držitele poučit o následcích neuposlechnutí. Pokud i tam věc nevydá, může být na základě rozhodnutí odňata. Odnětí musí být přítomna nezúčastněná osoba a musí být sepsán protokol.

Věc je možné zajistit také v rámci domovní prohlídky nebo prohlídky jiných prostor (oprávnění k prohlídce). Musí být přítomen znalec a sepsán protokol.

\subsubsection{Získání přístupu k vzdáleným datům}

Pokud jsou informace volně dostupné na internetu, je možné je použít a pořizovat z nich důkazní materiál. Jsou jisté pravidla a musí se sepsat protokol.

Pokud jsou data zabezpečena, lze k nim přistoupit pomocí přístupových údajů, které byly vydány dobrovolně v rámi výslechu či vysvětlení nebo jiným způsobem. Zabezpečená data jsou chápána jako písemnosti a záznamy uchovávané v soukromí. Bez přístupových údajů je možné přistoupit jen po povolení soudce a musí být sepsán protokol.

Pokud není možný odklad, lze k datům přistoupit ihned a podat žádost zpětně. Pokud povolení není do 48 hodin, data musí být zničena.

Data z mailů, chatů se berou jako odposlech.

\subsubsection{Získání dat od ISP}
Rozděluje se poskytovatel telekomunikačních služeb (data telekomunikačního provozu - telekomunikační tajemství, odposlech za daných podmínek) a poskytovatel služeb informační společnosti.

Dále se rozlišuje i charakter dat, různé typy dat vyžadují různé nástroje.

Data bez povinnosti mlčenlivosti musí vydat kdokoliv. Je povinné vyhovět dožádání orgánů činným v trestním řízení (často to co uživatel zveřejnil, informace o účtech, logy, metadata).

Zabezpečená data se berou jako záznamy uchovávané v soukromí.

Na dožádání můžou být data zachována po určitou dobu. Některé řízení trvá dlouho a hrozí tedy smazaná dat, na požádání poskytovatel data uchová.

\subsection{Detailněji rozebrané jednotlivé možnosti}
\subsubsection*{Obecná součinnost §8}
Státní orgány, fyzické a právní osoby a dálší relevantní subjekty mají povinnost vyhovovat na
dožádání - OČTŘ chce informaci a tak provede dožádání a daný subjekt by na tuto žádost měl
vyhovět - pokud existuje specifická právní úprava musí se použít ta - plus soudy pak řeší
proporcionalitu se zásahem do práv suběktu - zjišťování nejzákladnějších informací (obvykle v
rekognoskační fázy) - pro utajované informace je potřeba příkaz(souhlas) soudce (§8/5 "paragraf 8
odstavec 5)
\subsubsection*{Freezing §7b}
Vyžaduje to po nás Úmluva o kyberkriminalitě - 2 procesní nástroje
\begin{itemize}
    \item \textit{freezing} -- hrozí ztráta zníčení nebo pozměnění dat důležitých pro třestní řízení, lze nařídít osobě která je drží, aby je uchovala v nezměněné podobě pro potřeby dálšího vydání vyšetřovatelům (při phishingu zamrznutí nasbírané databáze)
    \item \textit{blocking} -- příkaz na provozovatele služby, aby zablokoval přístup užívatele k daný datům (max 90 dnů), poskytovately služby (ten kdo freezing provádí) musí být vysvětleno co má být zmraženo, proč a na jak dlouhou dobu
\end{itemize}

\subsubsection*{Odposlech a záznam telekomunikačního provozu §88}
při vedení TŘ pro zločin s odnětím svobody min. 8 let nebo pro vyjmenované TČ nebo pro
umyslné TČ k jehož stíhání nás zavazuje mezinárodní smlouva
\begin{itemize}
    \item může být vydán příkaz (úkon) pro zajištění obsahu telekomunikačního provozu (tekoucí data, e-maily telefonáty) - zajištěno \textbf{Útvarem zvláštních činností}
    \begin{itemize}
        \item mívají nainstalované zařízení, které umožnuje tento odposlech
        \item odposlech realizují ve spolupráci s operátorama a ti za to dostávají peníze
    \end{itemize}
    \item pokud nelze sledovaného účelu dosáhnout jinak nebo pokud by to bylo moc složité (preferují se
jiné úkony) - potřeba příkaz soudu - nebo i bez soudu se souhlasem uživatele - odposlech možný
vydat i dobudoucna (i když OČTŘ neví zda TČ probíhá) musí to však být stále dobře odůvodněné
\end{itemize}

\subsubsection*{Zajištění provozních a lokalizačních údajů §88a odst. 1}
metadata k tekoucí komunikaci - na základe příkazu soudu - pokud úmyslný TČ min 3 roky nebo
vyjmenované nebo TČ vyhlášený mezinározní smlouvou - pokud učelu nelze dosáhnout jinak -
nutno zajistit aby tyto údaje byly uloženy poskytovatelem - na to máme úpravu Zákona o
elektronických komunikacích

\subsubsection*{Data retention §97 - Zákon o elektronické komunikaci (ZoEK)}
ZoEK říká co jsou provozní a lokalizační údaje:
\begin{itemize}
    \item \textbf{metadata o komunikaci} - údaje by neměli nic říkat o přenášených datech (poskytovatel by třeba měl odfiltrovat data obsažená v URL třeba z formulářů)
    \item  poskytovatel je na základě data retention povinen uchovávat provozní a lokalizační údaje po dobu 6 měsíců od doby uskutečnení daného komunikačního provozu (za to poskytovatel dostává peníze), nesmí být poskytovatelem zneužita. Pokud nejsou po 6 měsících vyšetřovately požadovana musí je tekomunikační operátor zkartovat
\end{itemize}

\subsubsection*{Sledování osob a věcí 158d trestního řádu}
Pátrací prostředek - primárně určen pro zajištění operativních informací -> zjistit co se
stalo, ale jé na zjištění důkazů pro soud (tak tomu bylo původně)

Pokud chci zjištěné informace použít jako důkaz u soudu musím o tom vypracovat protokol - dnes se to používá i pro
zajištění el. dat jako důkazů. Postup:
\begin{enumerate}
    \item vyšetřovatel potřebuje data z uložiště -> zahájím sledování osob a věcí
    \item jako součinnost si vyžádám spolupráci od poskytovatele služby, který data uchovává -> a v rámci té součinnosti mi poskytně i ty data, která chci
    \item když udělám protokol o tom jak jsme získal danou spolupráci, kdo mě poskytnul součinnost, jak jsem postupoval a jaká data jsem získal -> mužu daná data použít jako el. důkaz u soudu
\end{enumerate}
O vydání příkazu o sledování osob a věcí rozhoduje státní zástupce prostřednictvím povolení - pokud jsou to ale soukromá nebo utajovaná data je potřeba předchozí povolení soudce - dá se udělat neodkladný úkol (získám telefon a neodkladně se kouknu na data uložená na vzdálené službě) pokud pětně bude soud souhlasit.

\subsubsection*{Domovní prohlídka a prohlídka jiných prostor}
Postup:
\begin{itemize}
    \item Soudce vydá příkaz k realizaci domovní prohlídky
    \item  policejní orgán tam provádí ohledání věcí relevantní k trestnímu stíhání - policejní orgán musí soudci dostatečně vysvětli proč se domnívá že v příslušných prostorách jsou veci důležité pro TŘ
    \item soudní zástupce sepíše podání na soud ve kterém žádá o vydání příkazu k domovní prohlídce ve které uvede co je cílem a proč a proč si myslím že to tam bude a oduvodním že to nejde jiným nástrojem
    \item  soud to posoudí a v rámci rozhodnutí pak vypíše informace na základě kterých se rozhodoval a tím oduvodní vydání daného příkazu - nedá se odvolat, ale stát ručí za škody způsobené nesprávným vydáním příkazu o domovní prohlídce
\end{itemize}
Procesní podmínky za kterých může být domovní prohlídka realizovaná:
\begin{itemize}
    \item přiměřenost -  odborná péče buď vyšetřovatel nebo znalec (když to dokážou ohledat na místě nemusí se to odvážet)
    \item potřeba přítomnosti majítele prostor nebo musí být aspoň informován o tom co se tam děje
    \item nezúčastněná osoba (někdo externí, soused) kdo zkontroluje že nedochází k porušení zákona při prohlídce
    \item vypracovává se protokol (videozáznam, foto) - všichni kdo se zučastní ho podepisují i nezúčastněná osoba
\end{itemize}
Aby přistihly zločince se zapnutím PC a přihlášením do vzdálené služby - dá se přizvat znalec pro specifickou
činnost se specifickým vybavením - musí se dodržovat specifické postupy aby nebyly důkazy znehodnocovány - zapečetění, zabalení do pytle za přítomnosti nezúčastněné osoby.

\subsubsection*{Výdání a odnětí věci}
Nástroj pro zajištění věci od člověka - každý má Ediční povinnost (na požádání musí vydat drženou věc) - OČTŘ ho vyzvou k předložení věci - pokud odmítně věc vydat může mu být odejmuta (pořádkové opatřené) stačí rozhodnutí státního zástupce - při odejmutí věci by měla být přítomna nezúčastněná osoba - pořizuje se protokol - osobě se dá potvrzení o odejmutí - pokud jsou na zařízení data o kterých je povinná mlčenlivost (utajované informace, advokátní tajemství) je specifický postup - pouze věci né data - ovšem pokud je vydán např telefon tak naněm může být provedena forenzní analýza a tedy je možno se dostat k datům uloženým na zařízení

\subsubsection*{Osobní prohlídka}
domnívám (jako OČTŘ) se že daná osoba má u sebe věc důležitou pro TŘ, ale nevím to jistě -> zahájím osobní prohlídku - na základě rozhodnutí soudu nebo státního zástupce - pokud je to neopakovatelný úkon můžu osobní prohlídku provéset i bez příkazu (zatknu osobu co utíká z místa činu a mám podezření že u sebe má zbraň se kterou páchal TČ) musím pak ale souhlas zpětně získat - je to zhojitelná vada neučinného důkazu - úkonu by měl předcházet předchozí výslech (jako u domovní prohlídky) což bych osobu měl požádat zda věč/předmět u sebe má a zda mi ho nevydá.

\subsubsection*{Ohledání věci}
pozorování a sbírání informací za účelem objasnění věci - protokol co bylo vypozorováno a jak - typicky při domovních a osobních prohlídkách - na místě najdu spuštěný počítač a na místě chci provést jeho ohledání - v takovém případě s kamerou nebo foťákem provádím záznam toho co jsem objevil.

Nemůžu provést obecné ohledání věci - např. když ohledávám na místě zaplé PC a bude tam probíhající komunikace a já bych ji chtěl sledovat (odposlech má větší zásah do práv), tak to nemůžu udělat jen na základě ohledání věci (generovalo by to neúčinný důkaz) - avšak policie si proto může předem připravit příkazy- například existují příkazy k domovní prohlídce, odposlechu a sledování osob a věcí a tím pádem může získávat všechno na místě.


\clearpage
\section{Typy a znaky skutkových podstat počítačových trestných činů}
Při spáchání trestného činu je nutné dokázat všechny znaky skutkové podstaty:

\subsection*{Subjekt}
Subjekt musí být právně způsobilý, což znamená, že je právně uznán jako subjekt, např. podle věku dítě není způsobilé (pachatel).

\subsection*{Objekt}
Sem patří právo (např. vlastnické právo) a právem chráněný zájem, který musí být právem uznán jako chráněný objekt (to co je chráněné TZ). \textit{Meč v hře je také uznán za chráněný, ne podle vlastnického práva, ale je jeho "majitelem" považován za cenný.}

\subsection*{Objektivní stránka}
Co bylo spácháno, jaké to mělo následky:
\begin{itemize}
    \item Škodlivý následek - škoda nastává v případě, že se dá empiricky vyčíslit, po prokázání mám nárok na náhradu, např. ušlý zisk. Nemateriální újma (odškodnění), např. psychická, újma dobré pověsti = nedají se empiricky vyčíslit.
    \item Protiprávní jednání - projev vůle, kterým subjekt porušuje příkazy nebo zákazy (právní povinnost) stanovené právní normou nebo vyplývající ze závazku, a s nímž se spojují právní následky (např. uložení trestu, náhrada škody, zánik práva, atd.).
    \item Kauzální nexus - příčinná souvislost mezi protiprávním jednáním a škodlivým následkem. Jeden proces (příčina) vyvolává druhý proces (následek nebo účinek).
\end{itemize}

\subsection*{Subjektivní stránka}
Týká se otázky, zda je možné prokázat zavinění. V trestním právu je potřebný úmysl:
\begin{itemize}
        \item úmysl - chtěl jsem jednat protiprávně:
        \begin{itemize}
             \item přímý - vražda je pouze na základě přímého úmyslu.
            \item nepřímý - chtěl jsem ho jen vylekat a nechtěně jsem ho přitom zabil, v tomto případě se nejedná o vraždu.
        \end{itemize}
        \item nedbalost:
        \begin{itemize}
            \item vědomá -- vím že se při mém jednání mohlo stát něco protiprávního ale neudělal jsem nic abych tomu zabránil.
            \item nevědomá -- nevěděl jsem, že mé jednání je protiprávní i když jsem to měl vědět.
            \item hrubá -- pokud profesionál v oboru ví o nějaké skutečnosti ale stejně na to nedbal.
        \end{itemize}
    \end{itemize}

\clearpage
\section{Subjektivní a objektivní odpovědnost}
\subsection*{Subjektivní odpovědnost}
\begin{itemize}
    \item vzniká v případě zaviněného porušení právní odpovědnosti
    \item zavinění - klíčový pojem
    \item se subj. odpovědností spojována exkulpace = vyvinění se $\longrightarrow$ možnost prokázat, že subjekt nezavinil vznik újmy přestože jsou splněny dané předpoklady
    \item musí se prokázat:
    \begin{itemize}
        \item subjekt - kdo a zda může být zodpovědný, pachatel
        \item objekt - právem chráněný zájem/právo (i virtuální věci)
        \item subjektivní stránka - zavinění, buď úmysl nebo nedbalost
        \item objektivní stránka - škodlivý následek, protiprávní následek, kauzální nexus (příčinná souvislost)
    \end{itemize}
\end{itemize}

\subsection*{Objektivní odpovědnost}
\begin{itemize}
    \item odpovědnost za protiprávní stav, událost
    \item nejde o odpovědnost za protiprávní jednání, ale za škodlivý následek
    \item možnost liberace = možnost zprostit se objektivní odpovědnosti; vyšší moc $\longrightarrow$ ke škodě by došlo i bez přičinění strany 
    \item pokud právem nelze  užít liberaci $\longrightarrow$ absolutní objektivní odpovědnost
    \item např. odpovědnost za vady, která vzniká nezávisle na zavinění zhotovitelem, příčina vady musí spočívat v něčem jiném, než jsou okolnosti způsobené objednatelem
    \item  u trestní odpovědnosti není objektivní odpovědnost - vždy se dokazuje zavinění 
\end{itemize} 

\subsection*{Prvky právní odpovědnosti}
\begin{itemize}
    \item = subjektivní stránka subjektivní odpovědnosti \\
\end{itemize}

\underline{Úmysl}
\begin{itemize}
    \item přímý
    \begin{itemize}
        \item ví, že se dopouští protipr. jednání, chce způsobit následek
        \item např. vražda
    \end{itemize}
    \item nepřímý
    \begin{itemize}
        \item ví, že se dopouští protipr. jednání, ale nemá v úmyslu způsobit škodlivý následek
        \item je možnost, že ví, že následek může nastat
        \item např. chtěl zranit, ale zabil \\
    \end{itemize}
\end{itemize}

\underline{Nedbalost}
\begin{itemize}
    \item vědomá
    \begin{itemize}
        \item ví, že se může něco stát a spoléhá na to, že se nic nestane
    \end{itemize}
    \item nevědomá
    \begin{itemize}
        \item neví, že porušuje právní normu a dopouští s protipr. jednání
        \item mohl a měl vědět, že se něco může stát
    \end{itemize}
\end{itemize}

\subsection*{Právní skutečnosti}
\begin{itemize}
    \item = okolnosti, se kterými právní norma spojuje vznik, změnu nebo zánik právního vztahu \\
\end{itemize}

\underline{Prokazatelné}
\begin{itemize}
    \item nutné prokázat, některé se neprokazují snadno
    \item lze prokázat i záporem (např. smrt člověka)
    \item korespondenční teorie pravdivosti - každou ze skutečností je třeba prokázat
    \item verifikace = prokázání důkazem X falzifikace = prokázání antiteze
    \item koherenční teorie - nemáme přímý ani falzifikovatelný důkaz, ale existují nepřímé důkazy (otisk prstů, svědectví)
    \item alibi může způsobit nekoherenci
    \item nepřímé důkazy - mají vztah s prokazovanou skutečností \\
\end{itemize}

\underline{Konsenzuální teorie}
\begin{itemize}
    \item skutečnost, že společnost se chová podle jistých nepsaných pravidel \\
\end{itemize}

\underline{Předpokládané} 
\begin{itemize}
    \item prokazuji předpoklad, skutečnost konstruuje právní norma
    \item domněnky - otcem je manžel matky, ale nemusí být
    \item fikce - skutečnost nenastala, ale právo říká, že nastala (např. fikce doručení - 10. den po oznámení) \\
\end{itemize}

\underline{Známé}
\begin{itemize}
    \item notoriety - všeobecně známé
    \item známé z úřední povinnosti (obchodní rejstřík)
    \item známé z rozhodovací činnosti - známé z praxe (sazba ČNB)
\end{itemize}
\clearpage


