\section{Komunikační řetězec, vrstvový model datového přenosu, základní operace při zpracování signálu u digitálního komunikačního systému. Úrovně signálu a vztažné hodnoty, absolutní a relativní úroveň, útlum, zisk, odstup signálu od šumu, výkonová spektrální hustota, přenosová kapacita kanálu.
}

\clearpage
\section{Princip zvyšování odolnosti přenášené zprávy proti chybám, informační poměr kódu, Hammingova vzdálenost, podmínky možnosti detekce a korekce chyb.}

\clearpage
\section{Vlastnosti ovlivňující návrh protichybového kódového systému. ARQ systémy.}

\clearpage
\section{Rozdělení protichybových kódů. Schéma realizace procesu kódování blokových kódů a stromových kódů. RM kódy, jejich základní parametry. Obecné blokové schéma kodéru turbokódu, význam jeho částí, dekódování turbokódů. Přehled možností používaných pro zabezpečení proti dlouhým shlukům chyb.}

\clearpage
\section{Kryptografické metody zabezpečení datových přenosů, architektura bezpečnosti, služby bezpečnosti, mechanizmy bezpečnosti.}

\clearpage
\section{Telekomunikační síť, struktura, způsoby komunikace, přenosové prostředky.}

\clearpage
\section{Metalická vedení, náhradní schéma homogenního vedení, primární parametry, sekundární parametry jednotky a vzájemné vztahy. Konstrukce symetrických kabelových vedení používaných v přístupové síti, DM a x čtyřky. Modely elektrických parametrů kabelových vedení určené pro simulaci DSL.}

\clearpage
\section{DSL systémy, vlastnosti, referenční konfigurace, typické uspořádání přípojky, možnosti využití. Základní charakteristiky jednotlivých systémů xDSL, IDSL, HDSL, SDSL, ADSL, VDSL, vlastnosti, možnosti použití. DSL použité kódy a modulace, 2B1Q, QAM, TCM,DMT, CAP.}

\clearpage
\section{Vliv rušení na provoz xDSL, kategorizace, dosažitelná přenosová rychlost, model přeslechů (NEXT, FEXT), princip výpočtu přeslechů. Spektrální vlastnosti DSL přenosových systémů, správa spektra, cíle, metody.}

\clearpage
\section{PLC systémy, princip, základní parametry, použité modulace, vazební členy, začlenění do sítě.}
