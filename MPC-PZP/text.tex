\section{Paralelismus – využití, paralelní technologie.}
\begin{itemize}
    \item Neplést \textit{parallelism} (využívá více jader, více úloh naráz) a~\textit{concurrency} (může se odehrávat i~na jediném jádře)
    \item Všechny paralelní programy jsou konkurentní, ne všechny konkurentní programy jsou paralelní
    \item Využití: není možné pokračovat s~trendem Moorova zákona\footnote{je to doopravdy jen pozorování, každých několik let se zdvojnásobil počet tranzistorů v~integrovaných čipech}, je nutné využít více jader pro zvýšení výkonnosti (zvýšení frekvence vyžaduje vyšší napájení, jak ale takový čip chladit?)
    \item Ne v~každé situaci je paralelismus výhodný
\end{itemize}

\subsection{paralelní technologie}
\begin{itemize}
    \item CPU (vícejádrový / vícevláknový (hyperthreading))
    \item GPU
\end{itemize}

\section{CPU – návrh paralelního programu, vlákna a procesy, synchronizace dat (Lock, RLock, Semaphore).}
Je potřeba zjistit úzká hrdla programu. Výpočty na sobě nezávislé mohou být paralelizované.

\subsection{Vlákna a~procesy}
Procesy mají vlastní paměťový prostor (kontext). Vlákna tento kontext mezi sebou sdílí, jeden proces může mít více vláken.

Procesy
\begin{itemize}
    \item Vytvoří se voláním \texttt{fork()}, \texttt{system()}, \texttt{popen()} nebo dalšími
    \item Minimální sdílení dat mezi rodičovským a \textit{child} procesem, běží na sobě nezávisle
\end{itemize}

Vlákna
\begin{itemize}
    \item Je to součást procesu, v~něm sdílí prostředky
    \item Jsou vhodné k~vykonání více operací nad sdílenými daty procesu
    \item Vlákna mohou být plně konkurentní
    \item Vhodná k~obsluze I/O (celý proces nemusí být blokován)
    \item Vlákna sdílí globální proměnné, ale mohou mít i~lokální proměnné nepřístupné jiným vláknům
    \item Je nutné je synchronizovat (mutual exclusion = mutex) pro zabránění souběhu
    \item Python je \underline{\textbf{stoopid}}, má GIL (Global Interpreter Lock) a~z~jediného procesu dokáže nechat běžet jen jediné vlákno. Má low-level (\texttt{thread}) a~high-level knihovnu (\texttt{threading})
    \item Stavy Python vláken:
    \begin{itemize}
        \item initial state
        \item runnable
        \item running
        \item blocked
        \item finished
    \end{itemize}
\end{itemize}

\subsection{Lock}
\begin{itemize}
    \item Jedná se o~zámek kritické sekce. 
    \item 2 stavy: zamčeno, odemčeno
    \item Pokud se zámek snaží získat vlákno:
    \begin{itemize}
        \item Pokud je zámek odemčený, zamkne ho vlákno pro sebe
        \item Pokud je zámek zamčený, vlákno je blokováno, dokud se zámek neuvolní
    \end{itemize}
\end{itemize}

\subsection{RLock}
\begin{itemize}
    \item Podobný jako Lock, ale R znamená \uv{reentrant}, další volání stejného vlákna na získání zámku toto vlákno nezablokuje
    \item Je potřeba zámek uvolnit stejně-krát jako byl získán
    \item Vhodné pro rekurzivní funkce
\end{itemize}

\subsection{Semaphore}
\begin{itemize}
    \item Umožňuje určitý počet přístupů
    \item Každé vlákno, které semafor získá, dekrementuje počítadlo
    \item Pokud vlákno chce získat přístup k~semaforu, s~počítadlem na nule, je blokované, dokud některé předchozí vlákno semafor neuvolní
    \item Vhodné pro limitování  počtu vláken, které mají přístup k~nějakému zdroji (třeba socket)
\end{itemize}


\section{GPU – vztahy mezi vlákny, bloky a mřížkou (grid), charakteristika Streaming Multiprocesoru.}
\subsection{Vlákno}
\begin{itemize}
    \item Vlákno provádí jedny instrukce (kernel) nad jednou částí dat
    \item Seskupení do Warpů: 32 jader má stejnou instrukci, ale různá data
    \item Warpy jsou lock-step (dělají stejné kroky ve stejný čas)
    \item SIMD (single instruction set, multiple data sets)
    \item Žadný context switching, každé vlákno má své registry, jejich velikost limituje počet vláken
    \item Má \texttt{threadIdx}, které může být 1D až 3D
\end{itemize}

\subsection{Bloky}
\begin{itemize}
    \item Skupina vláken, sdílí instrukce~a datas
    \item Vykonáván na jediném SM
    \item Má \texttt{blockIdx}, které může být 1D až 3D
    \item Mohou se umisťovat do fronty bloků, ty se pak rozdělují na SM
\end{itemize}

\subsection{Mřížka}
\begin{itemize}
    \item Skupina bloků
    \item Sdílet data mezi bloky lze jedině přes globální paměť (velká a~pomalá)
\end{itemize}

\subsection{Streaming Multiprocessor}
Popis Nvidia GV100 (jedna z~možných architektur)
\begin{itemize}
    \item Jádra FP32 (64 ks), FP64 (32 ks), INT32\dots
    \item Rozdělen na 4 procesní bloky, každý má čtvrtinu jader, registry, warp scheduler, L0 instruction cache\dots
    \item Warp (32 vláken) se posílá na vybraná jádra SM
    \item L1 cache (jedna na SM) rozdělena na instrukční a~datovou
    \item Grafické karty mají desítky SM
\end{itemize}

\section{GPU – paměti – rychlost, velikost, použití (vlákno, blok, mřížka).}
GPU má vlastní DRAM, je menší než DRAM v~PC, ale je do ní rychlejší přístup.

\subsection{Registry}
\begin{itemize}
    \item Nejrychlejší
    \item Nejmenší
    \item Přímo v~SM
    \item Přístup z~vlákna
    \item Rychlý přístup
    \item Stovky KB na SM (jsou jich tisíce, každý má 32 b)
    \item Vlákno může mít proměnlivý počet registrů, maximum dané architekturou
    \item Ukládá data přímo relevantní k~aktuální instrukci
\end{itemize}

\subsection{Lokální (L1) cache}
\begin{itemize}
    \item V~SM
    \item Obsahuje:
    \begin{itemize}
        \item Paměť sdílenou v~bloku (keyword \texttt{\_\_shared\_\_})
        \item Cache konstantní paměti
        \item Cache dat pro čtení
    \end{itemize}
    \item Řádky o~velikosti 128 B (vejde se 32 čísel o délce 32 b)
    \item Přístup z~bloku nebo z~vlákna
    \item Asi 80 cyklů potřeba pro přístup k~paměti
    \item Desítky kB (podle architektury)
    \item Obsahuje taky data pouze pro čtení
\end{itemize}

\subsection{L2 Cache}
\begin{itemize}
    \item Lze sdílet i~mezi bloky (přístup z~celé mřížky)
    \item 6 MB
    \item Stovky cyklů potřebných k~přístupu, až 1000 GB/s bandwidth
    \item \texttt{\_\_global\_\_} keyword
\end{itemize}

\subsection{DRAM}
\begin{itemize}
    \item Přístup z~celé mřížky
    \item Obsahuje:
    \begin{itemize}
        \item Globální paměť
        \item Konstantní paměť
        \item Paměť jen pro čtení
    \end{itemize}
    \item Jednotky až desítky GB
    \item Stovky cyklů potřebných k~přístupu, až 500 GB/s bandwidth
    \item DDR5
\end{itemize}

\subsection{Rozdíl mezi konstantní a~pamětí jen pro čtení}
\begin{itemize}
    \item Konstantní paměť se nakopíruje do každé cache stejně
    \item Paměť jen pro čtení může obsahovat velké množství dat, které se cachuje jen po částech (jen část vektoru relevantní pro blok)
\end{itemize}

\subsection{Použití cache}
\begin{itemize}
    \item Je potřeba navrhnout kernel tak, aby vyžadoval paměť co nejblíže (ideálně na jediném řádku cache)
    \item Při nedodržení (třeba při načítání každého tisícátého prvku vektoru) se výkon zpomalí kvůli cache misses
    \item Úplně ideální je, pokud jeden warp využívá data z~jediného řádku cache
\end{itemize}

\section{GPU – synchronizace, warp, divergence warpu, atomické operace.}
\subsection{Synchronizace}
\subsubsection{Synchronizace warpu a bloku}
\begin{itemize}
    \item Všechna vlákna warpu jsou synchronizovaná
    \item V bloku lze vlákna synchronizovat funkcí CUDA \texttt{\_\_syncthreads();}, tato funkce tvoří bariéru blokující, dokud všechna vlákna bloku nedoběhnou do této bariéry
    \item Může nastat deadlock, pokud tento statement není dosažen všemi vlákny kvůli větvení
    \item Pořadí bloků garantované není
    \item Další synchronizační funkce v bloku:
    \begin{itemize}
        \item \texttt{int \_\_syncthreads\_count(predicate)} -- kolik predikátů je pravdivých
        \item \texttt{int \_\_syncthreads\_and(predicate} -- vrátí \texttt{true}, pokud jsou všechny predikáty pravdivé
        \item \texttt{int \_\_syncthreads\_or(predicate)} -- vrátí \texttt{true}, pokud jsou aspoň některé predikáty pravdivé
    \end{itemize}
    \item taky existuje \texttt{\_\_threadfence\_block();} -- blokuje, dokud všechna vlákna v~bloku nevidí provedené zápisy v~globální a~sdílené paměti
\end{itemize}

\subsubsection{Synchronizace hostitele (PC) a zařízení (GPU)}
V kódu hosta volání \texttt{context.synchronize()}, zajistí, že všechny asynchronní úlohy jsou dokončeny. Blokuje na straně hosta.

\texttt{\_\_threadfence();} Blokuje vlákna v~bloku, dokud všechna z~nich nevidí změny ve sdílené paměti, nebo blokuje všechna vlákna, dokud nejsou změny v~globální paměti viditelné všem

\subsection{Warp}
\begin{itemize}
    \item 32 threadů
    \item Synchronizované \textit{lock-step} (stejná instrukce v~celém warpu)
    \item Každý blok má 1 nebo více Warpů
    \item Každý warp je plánovaný nezávisle na ostatních
    \item Scheduler preferuje warpy, které mají input přichystaný ke zpracování
    \item Protože warpy mají stejné instrukce, nastávají problémy při větvení
    \item Musí se vykonat obě větve pro celý warp, vlákna se pouze maskují ve větvi, která pro ně není relevantní
    \item Není potřeba warp voting, pokud je větvení jasné už z~kompilace (podmínka podle pevně dané hodnoty)
\end{itemize}

\subsection{Warp divergence}
\begin{itemize}
    \item = různá vlákna warpu vyžadují různé větve kódu
    \item Redukuje výkonnost paralelizace
    \item Nejhorší případ: zpomalení 32\(\times\) (pokud každé vlákno má jinou větev)
    \item Existují predikáty, fungují podobně jako if/else: \texttt{p = (x<0.0); p: z = x-2.0; !p:z= sqrt(x);}, pořád se jedná o~3~instrukce
    \item V rámci warpu také dochází k~\uv{warp voting}, každé vlákno avizuje, kterou větev bude vykonávat. Pokud všechna vlákna warpu vykonávají stejnou větev, je výpočet rychlejší (druhá větev se nepoužije vůbec)
\end{itemize}

\subsection{Atomické operace}
\begin{itemize}
    \item Při zápisu více vláken do jediné proměnné
    \item Čtení i zápis v jediné nedělitelné instrukci, zajistí integritu dat (nenastane souběh, při kterém by byla dříve přečtená hodnota změněna jiným vláknem před zápisem, což by přemazalo výstup druhého vlákna)
    \item \texttt{atomicAdd}
    \begin{itemize}
        \item první atribut je pointer, druhý hodnota k~přičtení
    \end{itemize}
    \item \texttt{atomicExch}
    \begin{itemize}
        \item Uloží hodnotu na adresu pointeru
    \end{itemize}
    \item \texttt{atomicMin}
    \item \texttt{atomicMax}
    \item \texttt{atomicCAS(int *addr, int compare, int val)} -- compare and swap, pokud je hodnota \texttt{compare} stejná jako ta na \texttt{addr}, nahraje tam místo toho \texttt{val} -- dá se použít k~atomickým zámkům
    \item A další
    \item Rychlé pro data ve sdílené paměti, pomalé pro globální paměť
\end{itemize}

\section{GPU - paralelní vzory, paralelní redukce, funkce shfl\_down\_sync(), asynchronní spouštění funkcí.}

\subsection{Paralelní vzory}
\begin{itemize}
    \item Jsou to stavební kameny pro složitější algoritmy
    \item Např.\,Map: aplikuje funkci na všechny prvky listu, snadno paralelizovatelné
    \item Další příklad: Scan (nový vektor, má o 1 méně hodnot, v~každém poli je součet hodnot původního pole po aktuální index)
\end{itemize}
\subsubsection{Redukce}
\begin{itemize}
    \item = Aplikování asociativního binárního operátoru na všechny prvky, redukování do jediné hodnoty
    \item +, *, max, min\dots
    \item Paralelní redukce: což takhle použít strom? -- redukce po párech (vznikne N/2 hodnot), opakovat, dokud nemám jedinou hodnotu
    \item Chtěli bychom dosáhnout toho, ať je celý wapr buď aktivní nebo neaktivní, chceme mitigovat warp divergenci
    \item Redukce na bloku vyžaduje synchronizaci
    \begin{verbatim}
__global__ void sum_reduction(float *input, float *block_results){
    extern __shared__ int sdata[];
    unsigned int i = blockIdx.x*blockDim.x + threadIdx.x;
    sdata[threadIdx.x] = input[i];
    __syncthreads();
    for (unsigned int stride = blockDim.x/2; stride > 0; stride/=2){
        unsigned int strided_i = threadIdx.x * 2 * stride;
        if (strided_i < blockDim.x){
            sdata[strided_i] += sdata[strided_i + stride]
        }
        __syncthreads();
    }
    if (threadIdx.x == 0) {
        block_results[blockIdx.x] = sdata[0];
    }
}
    \end{verbatim}
    \item Globální redukce je vhodná lokálními redukcemi, které jsou pak stejným sečteny atomickými operacemi
\end{itemize}

\subsection{shfl\_down\_sync()}
\subsubsection{Warp shuffle}
\begin{itemize}
    \item Pro přesun dat v~rámci jediného warpu
    \item Lze užít místo atomických operací, nevyžaduje sdílenou paměť, vlákna čtou registry jiných vláken
    \item Je nutné, aby bylo vlákno, ze kterého chci číst, aktivní, jinak se vrátí nedefinovaná hodnota
\end{itemize}
\subsubsection{Warp voting}
\begin{itemize}
    \item Aktivní vlákna mohou v~rámci warpu vyměňovat info o~podmínkách
    \item \texttt{int all\_sync(mask, condition)} -- True, pokud platí podmínka ve všech aktivních vláknech warpu
    \item \texttt{int any\_sync(mask, condition)} -- True, pokud platí podmínka v~aspoň jednom aktivním vláknu warpu
    \item \texttt{unsigned int ballot\_sync(mask, condition)} -- Vrací číslo, kde bity jsou dané hodnotami jednotlivých hodnot condition z~vláken
    \item Všechny tyto funkce blokují, ale jen aktivní vlákna
\end{itemize}
\subsubsection{Samotná funkce}
\begin{itemize}
    \item Argumenty:
    \begin{itemize}
        \item \texttt{unsigned mask} -- která vlákna se účastní
        \item \texttt{T var} -- proměnná
        \item \texttt{unsigned int delat} -- O kolik kroků dále se dívat na registry, z~těchto registrů čte hodnotu \texttt{var}. Pokud takový index neexistuje, vezme jen svou hodnotu registru
        \item \texttt{int width=warpSize}
    \end{itemize}
    \item Navrací hodnotu proměnné z~jiného registru
    \item Užití: třeba součet v~rámci warpu
\begin{verbatim}
#define FULL_MASK 0xffffffff
global void sum_warp_kernel_shfl_down(int *a)
{
    int local_sum = a[threadIdx.x + blockIdx.x*blockDim.x];
    for (int offset = WARP_SIZE / 2; offset>0; offset /= 2) {
        local_sum += shfl_down_sync(FULL_MASK, local_sum, offset);
    }
    if (threadIdx.x%32 == 0) {
        printf("Warp max is %d", local_sum)
    }
}
\end{verbatim}
\end{itemize}

\subsection{Asynchronní spouštění funkcí}
\begin{itemize}
    \item Synchronní blokuje
    \item Asynchronní neblokuje (spustím a jdu na další krok, ono to asi běží někde na pozadí, snad)
\end{itemize}

\subsubsection{Typy asynchronních funkcí}
\begin{itemize}
    \item Vzniklé v CPU pipeline
    \begin{itemize}
        \item napsáno synchronně, kompilátor vyhodnotil, že větve kódu na sebe nemají vliv a~vytvoří instrukce, které se překrývají pro maximální využití pipeline
    \end{itemize}
    \item CPU threading
    \begin{itemize}
        \item Podobná vlákna běží paralelně na procesoru, důležitá synchronizace
    \end{itemize}
    \item CUDA warp execution
    \begin{itemize}
        \item funkce ve stejném warpu jsou vzájemně synchronní
        \item warpy na SMP jsou spouštěny asynchronně
        \item synchronizace mezi warpy pomocí \texttt{syncthreads()}
        \item synchronní jsou:
        \begin{itemize}
            \item volání kernelu
            \item cudaMemcpy (device to device nebo host to device malých dat)
        \end{itemize}
        \item asynchronní volání jsou blokována:
        \begin{itemize}
            \item při \texttt{deviceSynchronize()}
            \item když má být spuštěn nový kernel
            \item Když se musí kopírovat data z~nebo na kartu
        \end{itemize}
    \end{itemize}
\end{itemize}

\subsubsection{CUDA streamy}
\begin{itemize}
    \item Jedná se o~asynchronní volání ve frontě na spuštění kartou
    \item V rámci streamu jsou operace seřazené, operace z~různých streamů řazené nejsou
    \item Lze kernelu přiřadit stream
    \item Podporují i~asynchronní přenos dat ve streamu (z~i~na kartu), ne pro defaultní stream
    \item Nejlepší je nastavit streamy tak, ať kopírovací enginy mohou přenášet data, zatímco kernely jsou vykonávány na SM. \(n\)-way concurrency znamená, kolik operací naráz karta provádí (5-way je třeba kopírování z~a~na kartu a~3 kernely k~tomu. Je nutné správně rozřadit operace do streamů)
    \item Všechny operace lze synchronizovat \texttt{cudaDeviceSynchronize()}
    \item Synchronizovat stream lze \texttt{cudaStreamSynchronize(stream)}
    \item synchronizaci lze provést i~pomocí eventů
\end{itemize}



\section{Apache Spark – vlastnosti, RDD, transformační funkce.}
\subsection{Vlastnosti}
\begin{itemize}
    \item Slovník:
    \begin{itemize}
        \item Job: Kód, co běží, vykonává funkci
        \item Stage: Každý job má stages, jsou buď typu map nebo reduce
        \item Task: Každá stage má jeden task na partition, je vykonaná na jednom stroji
        \item DAG: Directed Acyclic Graph, je to graf operací
        \item Executor: Proces vykonávající task
        \item Master: Stroj, na kterím běží Driver
        \item Slave: Stroj, na kterém běží executor
        \item Partition: část RDD, pracuje s~ní jeden executor
    \end{itemize}
    \item Komponenty:
    \begin{itemize}
        \item Driver: Tvoří SparkContext, komunikuje s cluster managerem, scheduluje tasks
        \item Executor(s): Spouštějí tasks, uloží data do paměti, na disk (Interakce s úložišti)
        \item Cluster Manager: Spravuje cluster, technologie: Mesos, YARN, Spark Standalone
    \end{itemize}
    \item Driver:
    \begin{itemize}
        \item Více částí (RDD Graph, DAGScheduler, Task Scheduler, Scheduler Backend), mění kód na jobs, ty vykoná na clusteru
    \end{itemize}
    \item Executor:
    \begin{itemize}
        \item Má block manager, ten získává bloky jak lokálně, tak vzdáleně
    \end{itemize}
    \item Spark Context:
    \begin{itemize}
        \item Reprezentuje spojení se Spark clusterem
        \item umožňuje vytvářet RDD, akumulátory a broadcast variables
    \end{itemize}
    \item DAG Scheduler:
    \begin{itemize}
        \item Vytvoří DAG stagů, předá je taskscheduleru
        \item Určí preferované místo pro tasks
    \end{itemize}
    \item Task Scheduler:
    \begin{itemize}
        \item Posílá tasks clusteru, spustí je, získá výsledky, retry při chybách\dots
    \end{itemize}
    \item Scheduler Backend
    \begin{itemize}
        \item Interface pro scheduling system (třeba YARN)
    \end{itemize}
    \item Block Manager
    \item SparkConf: objekt s~informacemi o~aplikaci
    \item Využívá lazy evaluation
\end{itemize}

\subsection{RDD}
= Resilient Distributed Dataset
\begin{itemize}
    \item Je typovaný
    \item Může být rozdělený na partitions
    \item Immutable
    \item Odolný proti chybám
    \item Paralelní
    \item Vytvoření z~úložiště nebo z~jiného RDD, uchovává informace o~rodičích (optimalizace pipeline nebo opakování při chybě)
    \item \texttt{collect()} vrátí data z~clusteru
    \item \texttt{glom()} metoda dělí data na partitions
    \item Persistence: dají se uložit na později, znovupoužít
\end{itemize}

Další typy:
\begin{itemize}
    \item Shared variables:
    \begin{itemize}
        \item Broadcast variables: ve všech nodes
        \item Accumulators: do nich se ukládají data z~různých míst (výsledek součtu apod.)
    \end{itemize}
    \item DataFrame (tabulka podobná SQL)
\end{itemize}

\subsection{Transformační funkce}
\begin{itemize}
    \item Aplikují transformace na každý prvek v~partition nebo na celou partition
    \item Reduce: agregační funkce (\texttt{groupBy}, \texttt{sortBy})
    \item Vytváří závislosti mezi RDD (to tvoří DAG)
    \item Umí znovu-vytvořit partitions
    \item Map: aplikuje funkci na RDD, z výsledků nový RDD (existuje i~flatMap -- každá funkce může vrátit sekvenci, ty se spojí)
    \item union, intersection, subtraction
    \item join
    \item Jsou úzké (pipelinable, každá partition závislá maximálně na jedné partition z~předchozího kroku) nebo široké (shuffle)
\end{itemize}

\section{Apache Spark – akumulátory, DataFrame, zpracování streamovaných dat.}
\subsection{Akumulátory}
\begin{itemize}
    \item Pouze asociativní a~komutativní operace
    \item Vestavěné implementace pro některé datové typy, lze doprogramovat další
    \item Vytváří se na driveru, ale akumulace se provádí na nodes clusteru
\end{itemize}
\subsection{DataFrame}
\begin{itemize}
    \item Immutable, distributed
    \item Pojmenované sloupce (tabulka)
    \item Zvládá CSV, JSON lépe než RDD
    \item Podporuje operace: transformace, akce
    \begin{itemize}
        \item Transformace jsou lazy (nepočítají se hned)
        \item DF se spočítá až po spuštění akce nad tímto DF
    \end{itemize}
    \item Persistence: dají se uložit do pamměti i~na disk
    \item Spark SQL, silně optimalizované, má optimizer Catalyst postavená na Scale
    \item přejmenování sloupců: \texttt{withColumn}
    \item Operace jak SQL
\end{itemize}

\subsection{Streaming}
\begin{itemize}
    \item Zachytávání dat v~reálném čase
    \item Sběr může být z~různých zdrojů, třeba Kafka, S3, multimédia\dots
    \item Streamy rozděleny na batches, i~výsledky v~batches
    \item DSTREAM = diskretizovaný, reprezentovaný jako sekvence RDD
    \item Je potřeba StreamingContext objekt, jen jeden může v~JVM existovat najednou
    \item Batches jsou RDD z~daného časového intervalu
    \item \texttt{updateStateByKey} -- podle klíčů updatuje stav (to definuje uživatel), používá funkce
    \item Window Streaming: posuvné okno RDDs, definuje 2 parametry: délku okna a~krok
    \item Streamy se dají joinnout vzájemně (podobně jako RDD)
    \item \texttt{forEachRDD} vykoná funkci nad každým RDD
    \item Structure Streaming: funguje s DataFrames, ne s~RDD
    \begin{itemize}
        \item microbatches, vše procesováno právě jednou
        \item Módy: kompletní (vše se uloží), append (jen data appended od poslední změny se uloží), update (jen updated data od posledního triggeru se uloží)
        \item Nedrží se vstupní data, jen aktuální stav
    \end{itemize}
    \item Vstupy: soubory, kafka, socket (ten je testovací)\dots
    \item output: Sink, má druhy:
    \begin{itemize}
        \item file
        \item kafka
        \item fireach
        \item console
        \item memory
    \end{itemize}
    \item Watermarking -- drží current event time, pokud se data moc zdrží, lze je zahodit
\end{itemize}



\section{Apache Spark - strojové učení, klasifikační algoritmy, klasterizace, četné vzory, TF-IDF.}
\section{Paralelní technologie – Apache Kafka, Nvidia Jetson, TPU.}