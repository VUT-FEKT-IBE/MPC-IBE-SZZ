\documentclass[
    % Velikost základního písma je 12 bodů
    12pt,
    % Formát papíru je A4
    a4paper,
    % Oboustranný tisk
    twoside,
    % Záložky a metainformace ve výsledném PDF budou v kódování unicode
    unicode,
]{article}

%%%%%%%%%%%%%%%%%%%%
% OBECNÉ NASTAVENÍ %
%%%%%%%%%%%%%%%%%%%%

% Kódování zdrojových souborů
\usepackage[utf8]{inputenc}
% Kódování výstupního souboru
\usepackage[T1]{fontenc}
% Podpora češtiny
\usepackage[czech]{babel}

% Geometrie stránky
\usepackage[
    % Vnitřní a vnější okraj
    hmargin={20mm,30mm},
    % Horní a dolní okraj
    vmargin={25mm,25mm},
    % Velikost zápatí
    footskip=17mm,
    % Vypnutí záhlaví
    nohead,
]{geometry}

% Zajištění kopírovatelnosti a prohledávanosti vytvořených PDF
\usepackage{cmap}
% Podmínky (pro použití v titulní straně)
\usepackage{ifthen}

%%%%%%%%%%%%%%%
% FORMÁTOVÁNÍ %
%%%%%%%%%%%%%%%

% Nastavení stylu nadpisů
\usepackage{sectsty}
% Formátování obsahů
\usepackage{tocloft}
\setcounter{tocdepth}{1}
% Odstranění mezer mezi řádky v seznamech
\usepackage{enumitem}
\setlist[itemize]{noitemsep}
% Sázení správných uvozovek pomocí '\enquote{}'
\usepackage{csquotes}
% Vynucení umístění poznámek pod čarou vespod stránky
\usepackage[bottom]{footmisc}
% Automatické zarovnání textu k předcházení vdov a parchantů
\usepackage[defaultlines=3,all=true]{nowidow}
% Zalomení části textu pokud není na současné stránce dost místa
\usepackage{needspace}
% Nastavení řádkování
\usepackage{setspace}
\onehalfspacing
% Změna odsazení odstavců
\setlength{\parskip}{1em}
\setlength{\parindent}{0em}

% Bezpatkové sázení nadpisů
\allsectionsfont{\sffamily}
% Změna formátování nadpisu a podnadpisů v Obsahu
\renewcommand{\cfttoctitlefont}{\Large\bfseries\sffamily}
\renewcommand{\cftsubsecdotsep}{\cftdotsep}

% Použití moderní/aktualizované sady písem
\usepackage{lmodern}

%%%%%%%%%%%
% NADPISY %
%%%%%%%%%%%

\usepackage{titlesec}

\titlespacing*{\section}{0pt}{10pt}{-0.2\baselineskip}
\titlespacing*{\subsection}{0pt}{0.2\baselineskip}{-0.2\baselineskip}
\titlespacing*{\subsubsection}{0pt}{0.2\baselineskip}{-0.2\baselineskip}
\titlespacing*{\paragraph}{0pt}{0pt}{1em}

%%%%%%%%%%
% ODKAZY %
%%%%%%%%%%

% Tvorba hypertextových odkazů
\usepackage[
    breaklinks=true,
    hypertexnames=false,
]{hyperref}
% Nastavení barvení odkazů
\hypersetup{
    colorlinks,
    citecolor=black,
    filecolor=black,
    linkcolor=black,
    urlcolor=blue
}

%%%%%%%%%%%%%%%%%%%%%%%%%%%
% OBRÁZKY, GRAFY, TABULKY %
%%%%%%%%%%%%%%%%%%%%%%%%%%%

% Vkládání obrázků
\usepackage{graphicx}
\usepackage{subfig}
% Nastavení popisů obrázků, výpisů a tabulek
\usepackage{caption}
\captionsetup{justification=centering}
% Grafy a vektorové obrázky
\usepackage{tikz}
\usetikzlibrary{shapes,arrows}
% Složitější tabulky
\usepackage{tabularx}
\usepackage{multicol}

% Sázení osamocených float prostředí v horní části stránky
\makeatletter
\setlength{\@fptop}{0pt plus 10pt minus 0pt}
\makeatother

%%%%%%%%%%%%%%
% MATEMATIKA %
%%%%%%%%%%%%%%

% Sázení matematiky a matematických symbolů ('\mathbb{}')
\usepackage{amsmath}
\usepackage{amssymb}
% Sázení fyzikálních veličin
\usepackage{siunitx}

% Náhrada za \mod a \pmod, které mají přehaně velký prostor před svou značkou
\newcommand{\Mod}[1]{\ \mathrm{mod}\ #1}
\newcommand{\Pmod}[1]{\ (\mathrm{mod}\ #1)}
% Náhrada za \forall, které kolem sebe prostor nemá žádný
\newcommand{\Forall}{\ \forall\ }

%%%%%%%%%%%%%%%%%
% ZDROJOVÉ KÓDY %
%%%%%%%%%%%%%%%%%

% Sazba zdrojových kódů
\usepackage[formats]{listings}
% Sazba zdrojových kódů
\usepackage[newfloat]{minted}
% Formátování 'minted' kódů
\usemintedstyle{pastie}
% Přepnutí prostředí 'code' do režimu výpisu kódu
\newenvironment{code}{\captionsetup{type=listing}}{}

%%%%%%%%%%%
% TITULKA %
%%%%%%%%%%%

\newcommand{\titulka}{
    \vspace*{2em}
    \begin{center}
        {\Huge \bfseries \subject}

        \vspace*{1em}

        {\Huge \bfseries \subjectname}

        \vspace*{2em}

        {\Large Otázky ke státnicím}
        
        \vspace*{1em}

        Magisterský obor Informační bezpečnost, FEKT VUT
        \\
        \url{https://github.com/VUT-FEKT-IBE/MPC-IBE-SZZ}
    \end{center}

    \vfill

    \begin{tabular}{ll}
        Text: & \authors \\
        Korektura: & \corrections \\
    \end{tabular}
    \hfill
    \today

    \clearpage
}
